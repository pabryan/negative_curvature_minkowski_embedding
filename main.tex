\documentclass[a4paper, 12pt]{amsart}

%\usepackage{etoolbox}
%\makeatletter
%\let\ams@starttoc\@starttoc
%\makeatother
%\makeatletter
%\let\@starttoc\ams@starttoc
%\patchcmd{\@starttoc}{\makeatletter}{\makeatletter\parskip\z@}{}{}
%\makeatother

%\usepackage[parfill]{parskip}
\usepackage{vmargin}
\usepackage[colorlinks=true,linkcolor=blue,citecolor=blue,urlcolor=blue]{hyperref}
\usepackage{bookmark}
\usepackage{amsthm,thmtools,amssymb,amsmath,amscd,amsfonts}
\usepackage{mathrsfs}
\usepackage{stmaryrd}


\usepackage[bibstyle=authoryear,citestyle=authoryear,backend=bibtex]{biblatex}
\bibliography{Bibliography}

\usepackage{fancyhdr}
\usepackage{esint}

\usepackage{enumerate}

\usepackage{pictexwd,dcpic}

\usepackage{graphicx}
\usepackage[utf8]{inputenc}

\declaretheorem[name=Theorem,numberwithin=section]{thm}
\declaretheorem[name=Remark,style=remark,sibling=thm]{rem}
\declaretheorem[name=Lemma,sibling=thm]{lemma}
\declaretheorem[name=Proposition,sibling=thm]{prop}
\declaretheorem[name=Definition,style=definition,sibling=thm]{defn}
\declaretheorem[name=Corollary,sibling=thm]{cor}
\declaretheorem[name=Assumption,style=remark,sibling=thm]{ass}
\declaretheorem[name=Example,style=remark,sibling=thm]{example}


\numberwithin{equation}{section}

\usepackage{cleveref}
\crefname{lemma}{Lemma}{Lemmata}
\crefname{prop}{Proposition}{Propositions}
\crefname{thm}{Theorem}{Theorems}
\crefname{cor}{Corollary}{Corollaries}
\crefname{defn}{Definition}{Definitions}
\crefname{example}{Example}{Examples}
\crefname{rem}{Remark}{Remarks}
\crefname{ass}{Assumption}{Assumptions}
\crefname{not}{Notation}{Notation}
\crefname{section}{Section}{Sections}

%Symbols
\renewcommand{\~}{\tilde}
\renewcommand{\-}{\bar}
\newcommand{\bs}{\backslash}
\newcommand{\cn}{\colon}
\newcommand{\sub}{\subset}

\newcommand{\N}{\mathbb{N}}
\newcommand{\R}{\mathbb{R}}
\newcommand{\Z}{\mathbb{Z}}
\renewcommand{\S}{\mathbb{S}}
\renewcommand{\H}{\mathbb{H}}
\newcommand{\C}{\mathbb{C}}
\newcommand{\K}{\mathbb{K}}
\newcommand{\Di}{\mathbb{D}}
\newcommand{\B}{\mathbb{B}}
\newcommand{\8}{\infty}

%Greek letters
\renewcommand{\a}{\alpha}
\renewcommand{\b}{\beta}
\newcommand{\g}{\gamma}
\renewcommand{\d}{\delta}
\newcommand{\e}{\epsilon}
\renewcommand{\k}{\kappa}
\renewcommand{\l}{\lambda}
\renewcommand{\o}{\omega}
\renewcommand{\t}{\theta}
\newcommand{\s}{\sigma}
\newcommand{\p}{\varphi}
\newcommand{\z}{\zeta}
\newcommand{\vt}{\vartheta}
\renewcommand{\O}{\Omega}
\newcommand{\D}{\Delta}
\newcommand{\G}{\Gamma}
\newcommand{\T}{\Theta}
\renewcommand{\L}{\Lambda}

%Mathcal Letters
\newcommand{\cL}{\mathcal{L}}
\newcommand{\cT}{\mathcal{T}}
\newcommand{\cA}{\mathcal{A}}
\newcommand{\cW}{\mathcal{W}}

%Mathematical operators
\newcommand{\INT}{\int_{\O}}
\newcommand{\DINT}{\int_{\d\O}}
\newcommand{\Int}{\int_{-\infty}^{\infty}}
\newcommand{\del}{\partial}

\newcommand{\inpr}[2]{\left\langle #1,#2 \right\rangle}
\newcommand{\abs}[1]{\left\lvert{#1}\right\rvert}
\newcommand{\fr}[2]{\frac{#1}{#2}}
\newcommand{\x}{\times}
\DeclareMathOperator{\Tr}{Tr}
\DeclareMathOperator{\Id}{Id}

\DeclareMathOperator{\dive}{div}
\DeclareMathOperator{\id}{id}
\DeclareMathOperator{\pr}{pr}
\DeclareMathOperator{\Diff}{Diff}
\DeclareMathOperator{\supp}{supp}
\DeclareMathOperator{\graph}{graph}
\DeclareMathOperator{\osc}{osc}
\DeclareMathOperator{\const}{const}
\DeclareMathOperator{\dist}{dist}
\DeclareMathOperator{\loc}{loc}
\DeclareMathOperator{\grad}{grad}
\DeclareMathOperator{\Ric}{Ric}
\DeclareMathOperator{\opRic}{\mathcal{R}ic}
\DeclareMathOperator{\Rm}{Rm}
\DeclareMathOperator{\Sc}{R}
\DeclareMathOperator{\Ein}{G}
\DeclareMathOperator{\opEin}{\mathcal{G}}
\DeclareMathOperator{\Sch}{P}
\DeclareMathOperator{\W}{\mathcal{W}}
\DeclareMathOperator{\inj}{inj}
\DeclareMathOperator{\adj}{adj}
\DeclareMathOperator{\Sym}{Sym}

%Environments
\newcommand{\Theo}[3]{\begin{#1}\label{#2} #3 \end{#1}}
\newcommand{\pf}[1]{\begin{proof} #1 \end{proof}}
\newcommand{\eq}[1]{\begin{equation}\begin{alignedat}{2} #1 \end{alignedat}\end{equation}}
\newcommand{\IntEq}[4]{#1&#2#3	 &\quad &\text{in}~#4,}
\newcommand{\BEq}[4]{#1&#2#3	 &\quad &\text{on}~#4}
\newcommand{\br}[1]{\left(#1\right)}

%Logical symbols
\newcommand{\Ra}{\Rightarrow}
\newcommand{\ra}{\rightarrow}
\newcommand{\hra}{\hookrightarrow}
\newcommand{\mt}{\mapsto}

%Names
\newcommand{\holder}{H\"older}

%Fonts
\newcommand{\mc}{\mathcal}
\renewcommand{\it}{\textit}
\newcommand{\mrm}{\mathrm}

%Spacing
\newcommand{\hp}{\hphantom}


%\parindent 0 pt

\protected\def\ignorethis#1\endignorethis{}
\let\endignorethis\relax
\def\TOCstop{\addtocontents{toc}{\ignorethis}}
\def\TOCstart{\addtocontents{toc}{\endignorethis}}


\newcommand{\note}[1]{\Rd {\bf[[ #1 ]]} \Bk}

\author{Paul Bryan and Mohammad N. Ivaki and Julian Scheuer}


%\author[P. Bryan]{Paul Bryan}
%\address{Department of Mathematics, Macquarie University NSW 2109, Australia}
%\email{\href{mailto:paul.bryan@uq.edu.au}{paul.bryan@uq.edu.au}}
%\urladdr{\href{http://pabryan.github.io}{http://pabryan.github.io/}}

%\author[M.N. Ivaki]{Mohammad N. Ivaki}
%\address{Department of Mathematics, University of Toronto, Ontario, M5S 2E4, Canada}
%\email{\href{mailto:m.ivaki@utoronto.ca}{m.ivaki@utoronto.ca}}

%\author[J. Scheuer]{Julian Scheuer}
%\address{Department of Mathematics, Columbia University New York, NY 10027, USA}
%\email{\href{mailto:jss2291@columbia.edu}{jss2291@columbia.edu}}
%\urladdr{\href{https://home.mathematik.uni-freiburg.de/scheuer/}{https://home.mathematik.uni-freiburg.de/scheuer/}}

\DeclareMathOperator{\Ob}{O}
\DeclareMathOperator{\opOb}{\mathcal{O}}
\DeclareMathOperator{\T}{T}
\DeclareMathOperator{\Dv}{D}
\DeclareMathOperator{\xcf}{\sigma}
\DeclareMathOperator{\dtxcf}{\xcf_{\operatorname{DT}}}
\renewcommand{\L}{\ensuremath{\operatorname{L}}}
\DeclareMathOperator{\dtrf}{\Ric_{\operatorname{DT}}}


\begin{document}

\title[Negative curvature Minkowski embedding]{Embedding negatively curved three-manifolds in Minkowski space}

\date{}

\dedicatory{}
\subjclass[2010]{58J35, 35K10, 58B20}
\keywords{Negative curvature, embedding, Minkowski, space-like}

\maketitle

\begin{abstract}
We consider the problem of embedding negatively curved three-manifolds into Minkowksi space.
\end{abstract}

\section{Introduction}
\label{sec:intro}

Let \((N, k)\) be a compact, Riemannian manifold with strictly negative curvature. Let \(\pi : (M, g) \to (N, k)\) be the Riemannian universal cover so that \(\pi : M \to N\) is a covering map with \(M\) simply connected and \(g = \pi^{\ast} k\). Let \(G\) denote the Deck transformation group of the cover and observe that \(g\) is invariant under \(G\). That is, \(G \leq \text{Diff}(M)\) is a group of diffeomorphisms of \(M\) and \(\varphi^{\ast} g = g\) for all \(\varphi \in G\) so that \(G\) acts by isometry on \((M, g)\). Then \(g\) induces a metric \(\bar{g}\) on the quotient \(M/G\) such that the isometry
\[
(M/G, \bar{g}) \underset{\simeq}{\to} (N, k)
\]
and the quotient map \(M \to M/G\) is just \(\pi\) under this identification. Then \((M/G, \bar{g})\) is a compact Riemannian quotient and we say \((M, g)\) is a co-compact Riemannian manifold.

\begin{rem}
The map \(\pi^{\ast}\) establishes an isomorphism of metrics on \(N\) with metrics on \(M\) invariant under the action of \(G\). Note also that all geometric invariants such as curvature are invariant under \(G\).
\end{rem}

Now, since \((N, k)\) has strictly negative sectional curvature, so does \((M, g)\) hence by the Cartan-Hadamard theorem, \(M \simeq \R^3\) is diffeomorphic to \(\R^3\) via the exponential map. In particular we may equip \(M\) with the hyperbolic metric, \(g_{\H}\) of constant, negative sectional curvature equal to \(-1\). Let us write \(G_{\H}\) for the isometry group of \((M, g_{\H})\).

On \(M\), there is a simple, smooth homotopy from \(g\) to \(g_{\H}\):
\[
h(t) = tg + (1-t)g_{\H}, \quad t \in [0, 1].
\]
This gives rise to the following very simple lemma:

\begin{lemma}
\label{lem:const_neg}
Let \((N, k)\) be a compact manifold of strictly negative sectional curvature. Then the following are equivalent:
\begin{enumerate}[i]
\item \label{enum:neg_met} \(N\) admits a metric of constant, negative sectional curvature,
\item \label{enum:deck_met} For \((M, g)\) the Riemannian universal cover of \((N, k)\), the hyperbolic metric \(g_{\H}\) is invariant under \(G\), the group of deck transformations of the cover \(\pi : M \to N\),
\item \label{enum:subgroup} The isometry group, \(G_{\H}\) of \((M, g_{\H})\) is a sub-group of \(G\),
\item \label{enum:homo_met} \(k\) is smoothly homotopic to a metric of constant, negative sectional curvature,
\item \label{enum:homo_deck} Every \(G\)-invariant metric \(g\) on \(M\) is smoothly homotopic to \(g_{\H}\) via a smooth \(G\)-invariant homotopy.
\end{enumerate}
\end{lemma}

\begin{proof}
Throughout, \(\pi : M \to N\) denotes the universal cover and \(G\) the group of deck transformations.

\begin{itemize}
\item \ref{enum:neg_met} is equivalent to \ref{enum:deck_met} is equivalent to \ref{enum:subgroup}:

Since \(\pi^{\ast}\) establishes an isomorphism of metrics on \(M\) invariant under \(G\) with metrics on \(N\) and since curvature is a local invariant, there is metric \(k_{\H}\) on \(N\) with constant sectional curvature, if and only if \(g^{\H}\) is invariant under \(G\) giving the first equivalence. The second equivalence follows since \(g_{\H}\) is invariant under \(G\) if and only \(G \leq G_{\H}\).

\item \ref{enum:neg_met} is equivalent to \ref{enum:homo_met}:

The condition \ref{enum:homo_met} obviously implies \ref{enum:neg_met}. Conversely, let \(g = \pi^{\ast} k\) denote the lifted metric to \(M\) which is then invariant under \(G\). Let \(k_{\H}\) denote the constant sectional curvature metric supposed to exist on \(N\). Then (after rescaling if necessary) the hyperbolic metric on \(M\) is given by \(g_{\H} = \pi^{\ast} k_{\H}\). In particular, \(g_{\H}\)  is invariant under \(G\) (which also proves \ref{enum:deck_met}). Thus for any \(t \in [0, 1]\), the metric \(g_t = t g + (1-t) g_{\H}\) is also invariant under \(G\) by linearity of the action \(\varphi^{\ast}\) for each \(\varphi \in G\) acting on metrics. Thus there exists a metric \(k_t\) on \(N\) such that \(g_t = \pi^{\ast} k_t\) and \(k_t\) gives the desired homotopy proving \ref{enum:homo_met}.

\item \ref{enum:homo_met} is equivalent to \ref{enum:homo_deck}:

Again the isomorphism \(\pi^{\ast}\) establishes an isomorphism between homotopies \(k_t\) on \(N\) and homotopies \(g_t\) invariant under \(G\) on \(M\).
\end{itemize}
\end{proof}

\begin{rem}
In condition \ref{enum:subgroup}, to say that \(G_{\H}\) is a sub-group of \(G\) here is via the concrete realisation (representation) of both groups acting on \(M\): the inclusion the inclusion map \(\iota : G_{\H} \to G\) is compatible with the actions of \(G_{\H}\) and \(G\) on \(M\) so that \(\iota(\varphi) \cdot x = \varphi \cdot x\) for every \(\varphi \in G_{\H}\) and \(x \in M\). Thus it is not enough for \(G_{\H}\) to be isomorphic to a subgroup of \(G\). The isomorphism must also be compatible with the actions.
\end{rem}

\begin{rem}
In the case of compact two-manifolds, the situation is completely understood. In this case, there is only one sectional curvature, the Gauss curvature \(K\) equal to half the scalar curvature \(R\). By the Gauss-Bonnet theorem, if \(K < 0\), \(0 < \int_N K d\mu = 2\pi(1-\lambda)\) where \(\lambda\) is the genus and hence \(\lambda > 1\). By the uniformisation theorem, all compact surfaces are classified by genus and moreover, each such surface admits a metric of constant Gauss curvature. Thus all compact surfaces of genus \(\lambda > 1\) admit a metric of constant negative sectional curvature, while the surfaces of genus \(\lambda = 0, 1\) do not even admit metrics of everywhere negative sectional curvature.

On the other hand, these hyperbolic surfaces (those with equipped with the constant negative sectional curvature metric) are precisely those presented as \(\H^2/\Gamma\) where \(\Gamma\) is a Fuschian subgroup of the isometry group \(G_{\H} = \text{PSL}(2, \R)\) of \(\H^2\). Then since all compact surfaces \(N\) with metrics \(k\) of strictly negative sectional curvature admit a constant curvature metric, all such surfaces are topologically quotients \(\H^2/\Gamma\). The lifted metric \(g = \pi^{\ast} k\) is then invariant under \(\Gamma\). Thus the deck transformation group of the Riemannian cover \((\H^2, g) \to (N = \H^2/\Gamma, k)\) is precisely the Fuschian subgroup \(\Gamma\) which is a subgroup of \(G_{\H}\). That is, the constant curvature metric is invariant under the deck transformation group \(\Gamma\) as in \Cref{lem:const_neg} and hence the homotopy \(h(t) = t g + (1-t)g_{\H}\) descends to the quotient \(N = M/\Gamma\) giving a homotopy \(k(t) = t k + (1-t)k_{\H}\) from \(k\) to the constant curvature metric \(k_{\H}\).
\end{rem}

The question of whether the conditions of \Cref{lem:const_neg} are satisfied are not easy to check but the lemma affords us with several possible approaches to the problem. Here we prove the following theorem giving a \emph{sufficient} condition for when \((N, k)\) admits a metric of constant, negative sectional curvature.

\begin{thm}[Integrability and constant negative sectional curvature]
\label{thm:intg_const_curv}
Let \((N, k)\) be a compact Riemannian three-manifold of strictly negative sectional curvature with the integrability condition that the tensor \(\alpha = \sqrt{\det \Ein} \Ein^{-1}\) is \emph{Codazzi}. Here \(\Ein\) denotes the Einstein tensor and to be Codazzi means that the three-tensor \(\nabla \alpha\) is totally symmetric.

Then \(N\) admits a metric of constant, negative sectional curvature.
\end{thm}

The theorem is in fact a corollary of another theorem we prove stating that with the same assumptions, the Riemannian universal cover of \((N, k)\) can be isometrically embedded into Minkowski space \(\R^{3,1}\).

\begin{thm}[Integrability implies embeddability]
\label{thm:intg_embed}
Let \((N, k)\) be a compact Riemannian three-manifold of strictly negative sectional curvature with the integrability condition that the tensor \(\alpha = \sqrt{\det \Ein} \Ein^{-1}\) is Codazzi.

Then the Riemannian universal cover \((M, g = \pi^{\ast} k)\) embeds isometrically into Minkowski space \(\R^{3,1}\).
\end{thm}

\begin{rem}
Necessarily, the embedding of \((M, g)\) is as a co-compact, spacelike hypersurface (i.e. the metric, \(g\) induced by the Lorentzian metric is positive definite). Since there are no compact, spacelike hypersurfaces in Minkowksi space, it is not possible to isometrically embed \((N, k)\).
\end{rem}

\begin{proof}
[Proof of \Cref{thm:intg_const_curv} assuming \Cref{thm:intg_embed}]

By \Cref{thm:intg_embed} we may embed \((N, k)\) into Minkowksi space as a co-compact, spacelike hypersurface. By the Gauss equation, since the second fundamental form is positive definite hence the embedding is in fact locally convex. By \cite[Theorem 1.1]{MR3344442}, the re-scaled flow Gauss curvature flow deforms \((N, k)\) smoothly to the hyperboloid at infinity with constant negative sectional curvature. Thus we obtain a homotopy from \((M, g)\) to \((M, g_{\H})\) and \Cref{lem:const_neg} gives the result.
\end{proof}

\begin{rem}
Under the Gauss curvature flow, as described in \cite[12. Application to the cross-curvature flow]{MR3344442}, the metric evolves by the Cross Curvature Flow introduced in \cite{MR2055396}. It is an open problem as to whether the Cross Curvature Flow deforms arbitrary negatively curved metrics to the constant curvature metric. A positive result would mean the integrability assumption in \Cref{thm:intg_const_curv} could be dropped.
\end{rem}

\section{Notation and preliminary results}
\label{sec:notation}

\subsection{Intrinsic geometry conventions}
\label{subsec:notation_intrinsic}

Given a metric \(g\), let \(\nabla = \nabla^g\) denote the Levi-Civita connection. The superscript will be dropped when the metric is clear from context. Let \(\inpr{\cdot}{\cdot}\) denote the inner-product on Minkowski space and \(D\) the corresponding Levi-Civita connection. Our conventions for the curvature tensor is
\[
\Rm(X, Y) Z = \nabla_X \nabla_Y Z - \nabla_Y \nabla_X Z - \nabla_{[X, Y]} Z
\]
and
\[
\Rm(X, Y, Z, W) = g(\Rm(X, Y) Z, W).
\]
The sectional curvature is then
\[
K(X \wedge Y) = \frac{\Rm(X, Y, Y X)}{\abs{X \wedge Y}_g^2}
\]
and \(g\) has constant sectional curvature \(K_0\) if and only if (note the minus sign!)
\[
\Rm = -\frac{K}{2} g \owedge g
\]
where
\[
\frac{1}{2} g \owedge g (X, Y, Z, W) = g(X, Z) g (Y, W) - g(X, W) g(Y, Z)
\]
denotes the Kulkarni-Nomizu product. Note that
\[
\frac{1}{2} g \owedge g (X, Y, Y, X) = -\abs{X \wedge Y}_g^2.
\]

We define the Ricci curvature
\[
\Ric(X, Y) = \Tr \Rm(\cdot, X) Y
\]
and scalar curvature
\[
\Sc = \Tr \mathcal{R}ic
\]
where \(\mathcal{R}ic\) is the self-adjoint endomorphism defined by \(\Ric(X, Y) = g(\mathcal{R}ic(X), Y)\). Equivalently, \(\Sc = \Tr_g \Ric := \Tr g^{\ast} \otimes \Ric\) is the full contraction and where \(g^{\ast}\) denotes the dual metric on \(T^{\ast}M\) induced by the metric \(g\) on \(TM\).

The Einstein tensor is defined by
\[
\Ein = \Ric - \frac{\Sc}{2} g
\]
We also have the Schouten tensor defined when \(n \geq 3\),
\[
\Sch = \frac{1}{n-2} \left(\Ric - \frac{\Sc}{2(n-1)}g \right).
\]
Obviously, given \(\Ric\) we obtain \(\Ein\) and \(\Sch\) but we may also recover \(\Ric\) from either of \(\Ein\) or \(\Sch\) so that any of the three determines the other two:
\[
\Ric = \Ein + \frac{1}{2-n} \Tr_g \Ein g
\]
and
\[
\Ric = (n-2) \Sch + \frac{n-2}{2n-3} \Tr_g \Sch g.
\]
Moreover, all three tensors are symmetric and simultaneously diagonalisable with an orthonormal basis of eigenvectors.

The Ricci decomposition of the curvature tensor can be written in terms of the three tensors as
\begin{align*}
\Rm &= -\Sch \owedge g + W \\
&= -\frac{1}{n-2} \left(\Ric - \frac{R}{2(n-1)} g\right) \owedge g + W \\
&= -\frac{1}{n-2} \left(\Ein  -\frac{1}{n-1} \Tr_g \Ein g\right) \owedge g + W
\end{align*}
where \(W = \Rm + \Sch \owedge g\) is the Weyl tensor. Note the first term in the sum on the right hand side of each line further decomposes into irreducible components but we don't need that here.

\subsection{Extrinsic geometry conventions}
\label{subsec:notation_extrinsic}

For an immersion \(F : M^n \to \R^{n,1}\) with \(M\) oriented, we define the second fundamental form by
\[
D_{F_{\ast} X} F_{\ast} Y = F_{\ast} \nabla_X Y + h(X, Y).
\]
The Gauss formula says that \(F_{\ast} \nabla_X Y = (D_{F_{\ast} X} F_{\ast} Y)^T\) is the tangential part of \(D_{F_{\ast} X} F{\ast} Y\) so that \(h(X, Y)\) is normal. The immersion is \emph{spacelike} if the induced metric \(g = F^{\ast} \inpr{\cdot}{\cdot}\) is positive definite, in which case the unit normal \(\nu\) is \emph{timelike}:
\[
\inpr{\nu}{\nu} = -1.
\]
Then we may write
\[
h(X, Y) = A(X, Y) \nu
\]
where
\[
A(X, Y) = -\inpr{D_{F_{\ast} X} F_{\ast} Y}{\nu} = \inpr{F_{\ast} Y}{D_{F_{\ast} X} \nu} = g(\W(X), Y)
\]
where \(\W = F_{\ast}^{-1} D\nu: TM \to TM\) denotes the Weingarten map.

Recalling that Minkowski space is flat, the Gauss equation for spacelike hypersurfaces reads
\[
\Rm(X, Y, Z, W) = -\inpr{h(X, Z)}{h(Y, W)} + \inpr{h(X, W)}{h(Y, Z)} = \frac{1}{2} A \owedge A (X, Y, Z, W).
\]
Since \(A(X, Y) = g(\W(X), Y)\) we have
\begin{align*}
g(\Rm(X, Y) Z, W) &= g(\W(X), Z) g(\W(Y), W) - g(\W(X), W) g(\W(Y), Z) \\
&= g(g(\W(X), Z) \W(Y) - g(\W(Y), Z) \W(X), W)
\end{align*}
so that
\[
\Rm(X, Y) Z = g(\W(X), Z) \W(Y) - g(\W(Y), Z) \W(X).
\]
Then
\[
\begin{split}
\Ric(X, Y) &= \Tr Z \mapsto \Rm(Z, X) Y \\
&= \Tr [Z \mapsto g(\W(Y), Z) \W(X)] - \Tr[Z \mapsto g(\W(X), Y) \W(Z)].
\end{split}
\]
The first map may be written \(Z \mapsto [g(\W(Y), \cdot) \otimes \W(X)] (Z)\) which has trace
\[
\Tr [Z \mapsto g(\W(Y), Z) \W(X)] = g(\W(Y), \cdot) (\W(X)) = g(\W(X), \W(Y)).
\]
The second is
\[
\Tr [Z \mapsto g(\W(X), Y) \W(Z)] = g(\W(X), Y) \Tr \W = H A(X, Y).
\]
Thus
\[
\Ric(X, Y) = g(\W(X),\W(Y)) - H A(X, Y).
\]
Since \(\W\) is self-adjoint we may write this as
\[
\Ric(X, Y) = g(\W^2(X) - H \W(X), Y)
\]
so that
\[
\Sc = \Tr [\W^2 - H\W] = \|A\|^2 - H^2.
\]

The Codazzi-Mainardi equation reads
\[
\nabla A (X; Y, Z) := \nabla_X A (Y, Z) = \nabla A(Y; X, Z).
\]
Of course, since \(A\) is already symmetric, \(\nabla A\) is automatically symmetric in the last two slots hence the Codazzi-Mainardi equation implies \(\nabla A\) becomes fully symmetric in all three slots.


\subsection{The three dimensional case}
\label{subsec:notation_threedim}

\subsubsection{Intrinsic geometry}

Specialising to the case \(n = 3\), we have
\[
\Sch = \Ric - \frac{R}{4} g \quad \text{and} \quad W = 0.
\]
Then the Ricci decomposition becomes
\[
\Rm = -\Sch \owedge g = -\Ric \owedge g + \frac{R}{4} g \owedge g = -\Ein \owedge g + \frac{\Tr_g G}{2} g \owedge g.
\]
Then we may express the sectional curvature
\begin{align*}
\abs{X \wedge Y} K(X \wedge Y) &= -2\Sch(X, Y) g(X, Y) + \Sch(X, X) g(Y, Y) + \Sch(Y, Y) g(X, X) \\
&= -2\Ric(X, Y) g(X, Y) + \Ric(X, X) g(Y, Y) + \Ric(Y, Y) g(X, X) - \frac{\Sc}{2} \abs{X \wedge Y} \\
&= -2\Ein(X, Y) g(X, Y) + \Ein(X, X) g(Y, Y) + \Ein(Y, Y) g(X, X) - \Tr_g \Ein \abs{X \wedge Y} \\
\end{align*}

In particular, if \(X, Y\) are orthonormal eigenvectors for \(\Ric\), then
\begin{align*}
K(X \wedge Y) &= \lambda^{\Sch}(X) + \lambda^{\Sch}(Y) \\
&= \lambda^{\Ric} (X) + \lambda^{\Ric} (Y) - \frac{\Sc}{2} \\
&= \lambda^{\Ein} (X) + \lambda^{\Ein} (Y) - \Tr_g \Ein \\
\end{align*}
where \(\lambda^{\Sch}(X)\) denotes the eigenvalue of \(\Sch\) corresponding to \(X\) and likewise for \(\lambda^{\Ric}\) and \(\lambda^{\Ein}\). In particular, if \(\{E_1, E_2, E_3\}\) is an orthonormal basis of eigenvectors, then the trace is simply the sum of the corresponding eigenvalues, hence for \(i \ne j\) and letting \(k\) denote the remaining index we have
\begin{align*}
K(E_i \wedge E_j) &= \lambda^{\Sch}_i + \lambda^{\Sch}_j \\
&= \lambda^{\Ric}_i  + \lambda^{\Ric}_j - \frac{1}{2} \sum_l \lambda^{\Ric}_l = \frac{1}{2}\left(\lambda^{\Ric}_i + \lambda^{\Ric}_j -\lambda^{\Ric}_k\right) \\
&= \lambda^{\Ein}_i + \lambda^{\Ein}_j - \sum_l \lambda^{\Ein}_l = -\lambda^{\Ein}_k.
\end{align*}
For this reason, on three manifolds, the Einstein tensor holds special significance with eigenvalues equal to the \emph{principal sectional curvatures}. Thus, although from the point of view of the curvature decomposition, it is more natural and simpler to work with \(\Sch\), from the point of view of sectional curvatures, it is more convenient to work with \(\Ein\). Of course the two are equivalent so we may choose whichever is more suited to our purposes which at present is to study negative sectional curvature so we work with \(\Ein\). In particular we note that:

\vfill
\emph{\(g\) has negative sectional curvature if and only if \(\Ein\) is positive definite.}
\vfill


\subsubsection{Extrinsic geometry}

The change of perspective from \(\Sch\) to \(\Ein\) is also quite useful when considering hypersurfaces in \(\R^{3,1}\). Recall the Gauss equation gave us
\[
\Ric(X, Y) = g(\W^2(X) - H \W(X), Y), \quad \Sc = \|A\|^2 - H^2.
\]
Note that
\[
\mathcal{R}ic (X) = \W^2(X) - H \W
\]
is simultaneously diagonalisable with \(\W\) and hence so too \(\Ein\) is simultaneously diagonalisable with \(A\). We have
\[
\Ein (X, Y) = g(\W^2(X), Y) - H A(X, Y) - \frac{\|A\|^2 - H^2}{2} g (X, Y).
\]
Denoting the eigenvalues of \(\W\) (i.e. the principal curvatures) by \(\kappa_i\), we then have with \(j \ne k\) the two distinct indices from \(i\),
\[
\begin{split}
-K(E_j \wedge E_k) = \lambda^{\Ein}_i &= \Ein(E_i, E_i) = \kappa_i^2 - \kappa_i \sum_l \kappa_l - \frac{\sum_l \kappa_l^2 - (\sum_l \kappa_l)^2}{2} \\
&= - \kappa_i \kappa_j - \kappa_i \kappa_k + \kappa_i \kappa_j + \kappa_i \kappa_k + \kappa_j \kappa_k \\
&= \kappa_j \kappa_k.
\end{split}
\]
Thus we see that the sectional curvatures are minus the product of the corresponding principal curvatures and \(\Ein\) is positive definite if and only if all the \(\kappa_i\) have the same sign. This is true either if all are negative or all are positive. By changing orientation \(\nu \mapsto -\nu\) if necessary we assume that \(\kappa_i > 0\).

\vfill
\emph{\(g\) has negative sectional curvature if and only if \(\Ein\) is positive definite if and only if \(A\) is positive definite (that is \(M\) embeds as a locally convex hypersurface).}
\vfill

Furthermore, the Gauss curvature is \(K = \det \W = \kappa_i \kappa_j \kappa_k\) and when \(A\) is positive definite, \(\W^{-1}\) is well defined, simultaneously diagonalisable with \(\W\) with eigenvalues \(\kappa_i^{-1}\) and hence
\[
\adj \W := \det \W \W^{-1} = \kappa_i \kappa_j \kappa_j \kappa_i^{-1} \delta_{il} = \kappa_j \kappa_k \delta_{il} = \lambda^{\Ein}_i \delta_{il} = \Ein.
\]
That is, the tensor \(\adj \W\) is \emph{intrinsic}. In fact, writing \(\Ein(X, Y) = g(\opEin (X), Y)\),
\[
\det \opEin = \lambda^{\Ein}_i \lambda^{\Ein}_j \lambda^{\Ein}_3 = \kappa_j \kappa_k \kappa_i \kappa_k \kappa_i \kappa_j = K^2.
\]
Then since \(\opEin^{-1}\) has eigenvalues \((\lambda_i^{\Ein})^{-1} = (\kappa_j \kappa_k)^{-1}\),
\[
\sqrt{\det \opEin} \opEin^{-1} = \kappa_i \kappa_j \kappa_k (\kappa_j \kappa_k)^{-1} \delta_{il} = \kappa_i \delta_{il} = \W
\]
we see that in fact \(\W\) is intrinsic! Therefore, defining \(\Ein^{-1}(X, Y) = g(\opEin^{-1} (X), Y)\) we have
\[
\alpha := \sqrt{\det \opEin} \Ein^{-1} = A
\]
and hence \(\alpha\) is Codazzi since by the Codazzi-Mainardi equations, \(A\) is Codazzi.

\begin{rem}
We see a strong rigidity statement that the \emph{extrinsic geometry} of embedded, spacelike hypersurfaces is completely determined by the \emph{intrinsic geometry}. Our main theorem is then the statement that if the intrinsic geometry satisfies \(\alpha\) is Codazzi, then in fact, the intrinsic geometry is induced by an spacelike embedding.
\end{rem}

\section{The embedding problem}
\label{sec:embedding}

\subsection{Reduction to graph case}
\label{subsec:embedding_graph}

\begin{lemma}
Let \(M^n \to \R^{n,1}\) be an immersed, spacelike hypersurface. Then \(M^n\) is embedded as a graph over the \(y^{n+1} = 0\) hyperplane.
\end{lemma}

\begin{proof}
Take any local parametrisation \(\varphi : U \subset \R^n \to \R^{n,1}\) of \(M\). For any non-zero \(X \in TU\) write
\[
d\varphi \cdot X = \bar{X} + X^{n+1} e_{n+1}
\]
with \(\bar{X} = \bar{X}^a \partial_a = \varphi^a_i X^i \partial_a \in \text{span} \{\partial_a : 1 \leq a \leq n\}\). Then since \(M\) is spacelike, we have
\[
0 < g(X, X) = \inpr{\bar{X} + X^{n+1} e_{n+1}}{\bar{X} + X^{n+1} e_{n+1}} = \|\bar{X}\|_{\R^n}^2 - (X^{n+1})^2.
\]
That is
\[
\|\bar{X}\|_{\R^n}^2 > (X^{n+1})^2 \geq 0.
\]

Now let \(\pi(y^1, \cdots, y^n, y^{n+1}) = (y^1, \cdots, y^n)\) denote the orthogonal projection onto the \(y^{n+1} = 0\) hyperplane and consider
\[
\pi \circ \varphi (x) = (\varphi^1(x), \cdots, \varphi^n(x)).
\]
Since \(d \pi\) is projection onto \(\text{span} \{\partial_a : 1 \leq a \leq n\}\) we obtain that
\[
d(\pi \circ \varphi) \cdot X = d\pi (\bar{X} + X^{n+1} e_{n+1}) = \bar{X}.
\]
Thus if \(X \ne 0\),
\[
\|d(\pi \circ \varphi) \cdot X\|_{\R^n}^2 = \|\bar{X}\|_{\R^n}^2 > 0
\]
and hence \(d(\pi \circ \varphi)\) is non-singular. By the inverse function theorem, locally \(\pi \circ \varphi\) is invertible and we find that
\[
\varphi \circ (\pi \circ \varphi)^{-1} (y^1, \cdots, y^n) = (y^1, \cdots, y^n, \varphi^{n+1} \circ (\pi \circ \varphi)^{-1} (y_1, \cdots, y_n))
\]
expresses \(M\) locally as a graph of the function \(\varphi^{n+1} \circ (\pi \circ \varphi)^{-1}\) over the \(y^{n+1} = 0\) plane. But this is true locally everywhere and since graphs are unique, \(M\) is globally a graph and hence also embedded.
\end{proof}

Thus we may assume any immersed, spacelike hypersurface is a graph. Now we may formulate the embedding problem as follows: Let \(g\) be a metric on \(\R^3\) with negative sectional curvature. Then \((\R^3, g)\) embeds isometrically into \(\R^{3,1}\) if and only if there exists a function \(u : \R^3 \to \R\) such that
\[
F : x \in \R^3 \mapsto (x, u(x)) \in \R^{3,1}
\]
is an isometric embedding \((\R^3, g) \to \R^{3,1}\). The induced metric on \(\R^3\) via the embedding \(F\) is
\[
F^{\ast} \inpr{\cdot}{\cdot} = \delta - Du \otimes Du
\]
where \(\delta\) denotes the Euclidean metric and \(Du = \partial_i dx^i\) the Euclidean differential. In coordinates the induced metric is
\[
\delta_{ij} - \partial_i u \partial_j u.
\]

The embedding problem is the equation \(\delta - Du \otimes Du = g\), or
\[
Du \otimes Du = \delta - g
\]
where \(\delta\) and \(g\) are given symmetric, positive definite bilinear forms. In coordinates, we have \(6\) equations in one unknown
\[
\partial_i u \partial_j u = \delta_{ij} - g_{ij}, \quad 1 \leq i \leq 3, \quad i \leq j \leq 3.
\]

This is an \emph{overdetermined} system. However, there is an a-priori necessary condition that \(g\) must satisfy in order that \(g = F^{\ast} \inpr{\cdot}{\cdot}\) can have a solution. This is that \(\alpha = \sqrt{\det \opEin} \Ein^{-1}\) is Codazzi. Thus if \(g\) is a metric such that \(\alpha\) is not Codazzi, then there are no solutions \(u\) to the embedding problem. Thus we restrict to case of metrics satisfying this condition which we think of as an integrability condition. The addition of this integrability condition allows us to reduce the number of independent equations for \(u\).

\subsection{Method of continuity}
\label{subsec:embedding_continuity}

The observation is that we can easily solve the embedding problem for both \(\delta\), the Euclidean metric and for \(g_{\H}\) the hyperbolic metric.
\[
u_{\delta} (x) = (x, 0), \quad u_{\H} (x) = \left(x, \sqrt{1 + \|x\|_{\R^3}^2}\right).
\]
The former simply embeds \(\R^3\) as the hyperplane \(\{x^4 = 0\}\) while the latter embeds \(\R^3\) as the upper sheet of the hyperboloid which is a well known model of \(g_{\H}\).

Now \Cref{thm:intg_embed} follows from the following theorem:

\begin{thm}
\label{thm:pde}
Let \(g\) be a metric on \(\R^3\) with negative sectional curvature, invariant under Lie group \(G\) such that \(\R^3/G\) is compact, and such that \(\alpha = \sqrt{\det P}P^{-1}\) is Codazzi. Moreover, assume that \(g(0) = \delta\). Then there exists a function \(u : \R^3 \to \R\) such that
\[
Du \otimes Du = \delta - g.
\]
\end{thm}

\begin{rem}
The condition \(g(0) = \delta\) is convenient for the proof and results in no loss of generality in proving the embedding \Cref{thm:intg_embed}. The hypotheses on \(g\) of of negative sectional curvature, co-compactness and \(\alpha\) Codazzi in \Cref{thm:intg_embed} are the same as in \Cref{thm:pde}. Then given such a \(g\), we may always perform a change of coordinates \(\varphi : \R^3 \to \R^3\) such that \(\varphi^{\ast} g (0) = \delta\). The hypotheses of either theorem are geometric hence preserved under this change of coordinates, so we may apply \Cref{thm:pde} to \(\varphi^{\ast} g\).

Letting \(u\) be any function obtained from \Cref{thm:pde} - so that \(Du \otimes Du = \varphi^{\ast} g - \delta\) - the embedding \(F(x) = (x, u(x))\) induces the metric \(F^{\ast} \inpr{\cdot}{\cdot} = \varphi^{\ast} g\). Then the embedding \(F \circ \varphi^{-1} (x)\) induces the metric
\[
(F \circ \varphi^{-1})^{\ast} \inpr{\cdot}{\cdot} = \varphi^{-1}_{\ast} F^{\ast} \inpr{\cdot}{\cdot} = \varphi^{-1}_{\ast} \varphi_{\ast} g = g
\]
thus providing the necessary embedding required by \Cref{thm:intg_embed}.
\end{rem}

\begin{proof}
Let
\[
g_t = t g + (1-t) g_{\H}
\]
where
\[
g_{\H} = \delta - Du_{\H} \otimes Du_{\H} = \delta - \frac{1}{1 + \|x\|_{\R^3}^2} \eta(x) \otimes \eta(x)
\]
is the hyperbolic metric realised as the graph of \(u_{\H}\) where \(\eta(x) = x^1 dx^1 + x^2 dx^2 + x^3 dx^3\) is the \emph{position one form}. In particular, \(g_{\H} (0) = \delta\) so that by the assumption \(g(0) = \delta\) we get that \(g_t(0) = \delta\) for all \(t \in [0, 1]\).

Now we seek to solve
\[
\Phi(u_t, t) := Du_t \otimes Du_t = \delta - g_t
\]
for \(u_t\) given \(t \in [0, 1]\). We can do this at \(t = 0\), since then \(g_0 = g_{\H}\) and
\[
\Phi(u_{\H}, 0) = D_{u_{\H}} \otimes D_{u_{\H}} = \delta - g_{\H}.
\]
The theorem will be proven if we can show that \(\Phi(u_t, t) = \delta - g_t\) has a solution for every \(t \in [0, 1]\) since then at \(t = 1\) we have \(g_1 = g\) and hence
\[
\Phi(u_1, 1) = Du_1 \otimes Du_1 = \delta - g
\]
as required.

So, let \(I \subset [0, 1]\) be the set of \(t \in [0, 1]\) such that there exists a \(u_t\) with \(F(u_t, t) = \delta\). Then as noted, \(0 \in I\) so that \(I\) is non-empty and the theorem follows by showing that \(I\) is both open and closed.

\begin{enumerate}
\item \(I\) is open:

As usual first we linearise. Define
\[
\begin{split}
L_u (v) &= \partial_s|_{s=0} F(u + sv) = \partial_s|_{s=0} \left(D(u + sv) \otimes D(u + sv)\right) \\
&= Du \otimes Dv + Dv \otimes Du = 2 \Sym (Du \otimes Dv).
\end{split}
\]
In passing, observe that choosing \(g_t = t g + (1-t)\delta\) would now be problematic since in that case any solution \(u_{\delta} \equiv \text{const}\) to \(F(u_{\delta}, 0) = 0\) has vanishing linearisation.

At \(t=0\), we have the linear operator
\[
L_0 = L_{\H} : v \mapsto 2 \Sym Du_{\H} \otimes Dv = \frac{2}{\sqrt{1 + \abs{x}^2}} \Sym \eta(x) \otimes Dv(x)
\]
with \(\eta\) the position one form as above. This is a fibre-wise linear map acting on co-vectors
\[
L(x) : \omega \in T_x^{\ast} \R^3 \mapsto \frac{1}{\sqrt{1 + \abs{x}^2}} \Sym \eta(x) \otimes \omega \in T^{\ast}_x \R^3 \odot T^{\ast}_x \R^3
\]
which is injective for \(x \ne 0\) but degenerates to the zero map \(0\) at \(x = 0\). Note that for \(x \ne 0\), \(\omega \in \operatorname{Ker} L(x)\) if and only if \(\eta(x) \otimes \omega\) is in the kernel of \(\Sym\) if and only if \(\eta(x) \otimes \omega = - \omega \otimes \eta(x)\). Writing \(\omega = \omega_i dx^i\) and \(\eta(x) = x_i dx^i\) with the Einstein summation convention, we have
\[
\Sym \eta(x) \otimes \omega = x_i dx^i \otimes \omega_j dx^j = x_i \omega_j dx^i \otimes dx^j.
\]
Supposing that \(x \ne 0\), we have without loss of generality, \(x_1 \ne 0\) so that \(\Sym \eta(x) \otimes \omega = 0\) implies \(x_1 \omega_j = 0\) for each \(j\) and hence \(\omega = 0\). In other words, the symmetric product is a perfect pairing and hence has trivial kernel hence \(L(x)\) is injective as claimed whenever \(x \ne 0\).

In order to apply the implicit function theorem and conclude that \(\Phi(u_t, t) = \delta - g\) has a solution for \(t \in [0, \epsilon)\) for some \(\epsilon > 0\) we need to know that \(\Phi\) maps into the space of bilinear forms \(\mathcal{B}\) of the form \(\delta - g\) where \(g\) satisfies the hypotheses of the theorem and that \(L_0\) is an isomorphism between the tangent space to functions \(u\) with \(Du (0) = 0\) at \(u_{\H}\) and the tangent space to bilinear forms \(\mathcal{B}\) at \(\delta - g_{\H}\).

Now we must define the appropriate spaces. {\color{red} Figure out the appropriate topology later}.

Let
\[
A = \{u \in C^{\infty}(\R^3 \to \R) : Du(0) = 0\}.
\]
Then if \(u_t\) is a path in \(A\), \(v := \partial_t|_{t=0} u_t \in A\) also. Conversely, if \(v \in A\), then \(u + t v \in A\) so that for each \(u \in A\),
\[
T_u A \simeq A.
\]
In fact \(A\) is clearly a linear subspace from which \(T_u A \simeq A\) follows.

Let
\[
B = \Phi (A) = \{Du \otimes Du : u \in A\} = \{\omega \otimes \omega : d \omega = 0, \omega(0) = 0\}
\]
with the last equality following by the Poincar\'e lemma on \(\R^3\). That is, \(B\) is the space of simple, symmetric, quadratic forms built from exact one forms that vanish at the origin.

If \(\omega_t \otimes \omega_t\) is a path in \(B\) with \(\nu = \partial_t|_{t=0} \omega_t\) , then \(\mu := \partial_t|_{t=0} \omega_t \otimes \omega_t = 2 \Sym \omega_t \otimes \nu\). Notice that \(d\omega_t = 0 \Rightarrow d\nu = 0\). Conversely, given \(\eta\) with \(d \eta = 0\), \((\omega + t \eta) \otimes (\omega + t \eta) \in B\) with \(\mu =  2 \Sym \omega_t \otimes \nu\). Thus the tangent space to \(B\) is then,
\[
T_{\omega \otimes \omega} B = \{2 \Sym D\omega \otimes \nu : d\nu = 0, \nu(0) = 0\}.
\]

\item \(I\) is closed:


\end{enumerate}
\end{proof}

\printbibliography

\end{document}
