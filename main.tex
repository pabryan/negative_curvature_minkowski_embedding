\documentclass[a4paper, 12pt]{amsart}

%\usepackage{etoolbox}
%\makeatletter
%\let\ams@starttoc\@starttoc
%\makeatother
%\makeatletter
%\let\@starttoc\ams@starttoc
%\patchcmd{\@starttoc}{\makeatletter}{\makeatletter\parskip\z@}{}{}
%\makeatother

%\usepackage[parfill]{parskip}
\usepackage{vmargin}
\usepackage[colorlinks=true,linkcolor=blue,citecolor=blue,urlcolor=blue]{hyperref}
\usepackage{bookmark}
\usepackage{amsthm,thmtools,amssymb,amsmath,amscd,amsfonts}
\usepackage{mathrsfs}
\usepackage{stmaryrd}


\usepackage[bibstyle=authoryear,citestyle=authoryear,backend=bibtex]{biblatex}
\bibliography{Bibliography}

\usepackage{fancyhdr}
\usepackage{esint}

\usepackage{enumerate}

\usepackage{pictexwd,dcpic}

\usepackage{graphicx}
\usepackage[utf8]{inputenc}

\declaretheorem[name=Theorem,numberwithin=section]{thm}
\declaretheorem[name=Remark,style=remark,sibling=thm]{rem}
\declaretheorem[name=Lemma,sibling=thm]{lemma}
\declaretheorem[name=Proposition,sibling=thm]{prop}
\declaretheorem[name=Definition,style=definition,sibling=thm]{defn}
\declaretheorem[name=Corollary,sibling=thm]{cor}
\declaretheorem[name=Assumption,style=remark,sibling=thm]{ass}
\declaretheorem[name=Example,style=remark,sibling=thm]{example}


\numberwithin{equation}{section}

\usepackage{cleveref}
\crefname{lemma}{Lemma}{Lemmata}
\crefname{prop}{Proposition}{Propositions}
\crefname{thm}{Theorem}{Theorems}
\crefname{cor}{Corollary}{Corollaries}
\crefname{defn}{Definition}{Definitions}
\crefname{example}{Example}{Examples}
\crefname{rem}{Remark}{Remarks}
\crefname{ass}{Assumption}{Assumptions}
\crefname{not}{Notation}{Notation}
\crefname{section}{Section}{Sections}

%Symbols
\renewcommand{\~}{\tilde}
\renewcommand{\-}{\bar}
\newcommand{\bs}{\backslash}
\newcommand{\cn}{\colon}
\newcommand{\sub}{\subset}

\newcommand{\N}{\mathbb{N}}
\newcommand{\R}{\mathbb{R}}
\newcommand{\Z}{\mathbb{Z}}
\renewcommand{\S}{\mathbb{S}}
\renewcommand{\H}{\mathbb{H}}
\newcommand{\C}{\mathbb{C}}
\newcommand{\K}{\mathbb{K}}
\newcommand{\Di}{\mathbb{D}}
\newcommand{\B}{\mathbb{B}}
\newcommand{\8}{\infty}

%Greek letters
\renewcommand{\a}{\alpha}
\renewcommand{\b}{\beta}
\newcommand{\g}{\gamma}
\renewcommand{\d}{\delta}
\newcommand{\e}{\epsilon}
\renewcommand{\k}{\kappa}
\renewcommand{\l}{\lambda}
\renewcommand{\o}{\omega}
\renewcommand{\t}{\theta}
\newcommand{\s}{\sigma}
\newcommand{\p}{\varphi}
\newcommand{\z}{\zeta}
\newcommand{\vt}{\vartheta}
\renewcommand{\O}{\Omega}
\newcommand{\D}{\Delta}
\newcommand{\G}{\Gamma}
\newcommand{\T}{\Theta}
\renewcommand{\L}{\Lambda}

%Mathcal Letters
\newcommand{\cL}{\mathcal{L}}
\newcommand{\cT}{\mathcal{T}}
\newcommand{\cA}{\mathcal{A}}
\newcommand{\cW}{\mathcal{W}}

%Mathematical operators
\newcommand{\INT}{\int_{\O}}
\newcommand{\DINT}{\int_{\d\O}}
\newcommand{\Int}{\int_{-\infty}^{\infty}}
\newcommand{\del}{\partial}

\newcommand{\inpr}[2]{\left\langle #1,#2 \right\rangle}
\newcommand{\abs}[1]{\left\lvert{#1}\right\rvert}
\newcommand{\fr}[2]{\frac{#1}{#2}}
\newcommand{\x}{\times}
\DeclareMathOperator{\Tr}{Tr}
\DeclareMathOperator{\Id}{Id}

\DeclareMathOperator{\dive}{div}
\DeclareMathOperator{\id}{id}
\DeclareMathOperator{\pr}{pr}
\DeclareMathOperator{\Diff}{Diff}
\DeclareMathOperator{\supp}{supp}
\DeclareMathOperator{\graph}{graph}
\DeclareMathOperator{\osc}{osc}
\DeclareMathOperator{\const}{const}
\DeclareMathOperator{\dist}{dist}
\DeclareMathOperator{\loc}{loc}
\DeclareMathOperator{\grad}{grad}
\DeclareMathOperator{\Ric}{Ric}
\DeclareMathOperator{\opRic}{\mathcal{R}ic}
\DeclareMathOperator{\Rm}{Rm}
\DeclareMathOperator{\Sc}{R}
\DeclareMathOperator{\Ein}{G}
\DeclareMathOperator{\opEin}{\mathcal{G}}
\DeclareMathOperator{\Sch}{P}
\DeclareMathOperator{\W}{\mathcal{W}}
\DeclareMathOperator{\inj}{inj}
\DeclareMathOperator{\adj}{adj}
\DeclareMathOperator{\Sym}{Sym}

%Environments
\newcommand{\Theo}[3]{\begin{#1}\label{#2} #3 \end{#1}}
\newcommand{\pf}[1]{\begin{proof} #1 \end{proof}}
\newcommand{\eq}[1]{\begin{equation}\begin{alignedat}{2} #1 \end{alignedat}\end{equation}}
\newcommand{\IntEq}[4]{#1&#2#3	 &\quad &\text{in}~#4,}
\newcommand{\BEq}[4]{#1&#2#3	 &\quad &\text{on}~#4}
\newcommand{\br}[1]{\left(#1\right)}

%Logical symbols
\newcommand{\Ra}{\Rightarrow}
\newcommand{\ra}{\rightarrow}
\newcommand{\hra}{\hookrightarrow}
\newcommand{\mt}{\mapsto}

%Names
\newcommand{\holder}{H\"older}

%Fonts
\newcommand{\mc}{\mathcal}
\renewcommand{\it}{\textit}
\newcommand{\mrm}{\mathrm}

%Spacing
\newcommand{\hp}{\hphantom}


%\parindent 0 pt

\protected\def\ignorethis#1\endignorethis{}
\let\endignorethis\relax
\def\TOCstop{\addtocontents{toc}{\ignorethis}}
\def\TOCstart{\addtocontents{toc}{\endignorethis}}


\newcommand{\note}[1]{\Rd {\bf[[ #1 ]]} \Bk}

\author{Paul Bryan and Mohammad N. Ivaki and Julian Scheuer}


%\author[P. Bryan]{Paul Bryan}
%\address{Department of Mathematics, Macquarie University NSW 2109, Australia}
%\email{\href{mailto:paul.bryan@uq.edu.au}{paul.bryan@uq.edu.au}}
%\urladdr{\href{http://pabryan.github.io}{http://pabryan.github.io/}}

%\author[M.N. Ivaki]{Mohammad N. Ivaki}
%\address{Department of Mathematics, University of Toronto, Ontario, M5S 2E4, Canada}
%\email{\href{mailto:m.ivaki@utoronto.ca}{m.ivaki@utoronto.ca}}

%\author[J. Scheuer]{Julian Scheuer}
%\address{Department of Mathematics, Columbia University New York, NY 10027, USA}
%\email{\href{mailto:jss2291@columbia.edu}{jss2291@columbia.edu}}
%\urladdr{\href{https://home.mathematik.uni-freiburg.de/scheuer/}{https://home.mathematik.uni-freiburg.de/scheuer/}}

\DeclareMathOperator{\Ob}{O}
\DeclareMathOperator{\opOb}{\mathcal{O}}
\DeclareMathOperator{\T}{T}
\DeclareMathOperator{\Dv}{D}
\DeclareMathOperator{\xcf}{\sigma}
\DeclareMathOperator{\dtxcf}{\xcf_{\operatorname{DT}}}
\renewcommand{\L}{\ensuremath{\operatorname{L}}}
\DeclareMathOperator{\dtrf}{\Ric_{\operatorname{DT}}}


\begin{document}

\title[Negatively Curved Three Manifolds]{Negatively Curved Three-Manifolds, Hyperbolic Metrics, Isometric Embedding In Minkowski Space And The Cross Curvature Flow}

\date{}

\dedicatory{}
\subjclass[2010]{58J35, 35K10, 58B20}
\keywords{Negative curvature, embedding, Minkowski, space-like}

\begin{abstract}
We consider the problem of embedding negatively curved three-manifolds into Minkowksi space.
\end{abstract}

\maketitle

\section{Introduction}
\label{sec:intro}
Let \((N, k)\) be a compact, Riemannian manifold with strictly negative curvature. Let \(\pi\colon (M, g) \to (N, k)\) be the Riemannian universal cover so that \(\pi : M \to N\) is a covering map with \(M\) simply connected and \(g = \pi^{\ast} k\). Let \(G\) denote the Deck transformation group of the cover and observe that \(g\) is invariant under \(G\). That is, \(G \leq \text{Diff}(M)\) is a group of diffeomorphisms of \(M\) and \(\varphi^{\ast} g = g\) for all \(\varphi \in G\) so that \(G\) acts by isometry on \((M, g)\). Then \(g\) induces a metric \(\bar{g}\) on the quotient \(M/G\) such that
\[
(M/G, \bar{g}) \underset{\simeq}{\to} (N, k)
\]
is an isometry and the quotient map \(M \to M/G\) is just \(\pi\) under this identification. Then \((M/G, \bar{g})\) is a compact Riemannian quotient and we say \((M, g)\) is a co-compact Riemannian manifold.
\begin{rem}
The map \(\pi^{\ast}\) establishes an isomorphism of metrics on \(N\) with metrics on \(M\) invariant under the action of \(G\). Note also that all geometric invariants such as curvature are invariant under \(G\).
\end{rem}
Now, since \((N, k)\) has strictly negative sectional curvature, so does \((M, g)\) hence by the Cartan-Hadamard theorem, \(M \simeq \R^3\) is diffeomorphic to \(\R^3\) via the exponential map. In particular we may equip \(M\) with the hyperbolic metric, \(g_{\H}\) of constant, negative sectional curvature equal to \(-1\). Let us write \(G_{\H}\) for the isometry group of \((M, g_{\H})\).

On \(M\), there is a simple, smooth homotopy from \(g\) to \(g_{\H}\):
\[
h(t) = tg + (1-t)g_{\H}, \quad t \in [0, 1].
\]
This gives rise to the following lemma:
\begin{lemma}
\label{lem:const_neg}
Let \((N, k)\) be a compact manifold of strictly negative sectional curvature. Then the following statements are equivalent:
\begin{enumerate}[(i)]
\item \label{enum:neg_met} \(N\) admits a metric of constant, negative sectional curvature.
\item \label{enum:deck_met} \(g_{\H}\) is invariant under \(G\).
\item \label{enum:subgroup} G is a subgroup of \(G_{\H}\).
\item \label{enum:homo_met} \(k\) is smoothly homotopic to a metric of constant, negative sectional curvature.
\item \label{enum:homo_deck} Every \(G\)-invariant metric \(g\) on \(M\) is smoothly homotopic to \(g_{\H}\) via a smooth \(G\)-invariant homotopy.
\end{enumerate}
\end{lemma}
\begin{proof}

(i)\(\Rightarrow\)(ii)
The pullback of a constant curvature metric is the hyperbolic one, which is then \(G\)-invariant.

(ii)\(\Rightarrow\)(iii)
Clear, since \(G_{\H}\) is the whole isometry group.

(iii)\(\Rightarrow\)(iv)
Since \(g\) and \(g_{\H}\) are invariant under G, so is \(h(t)\), which in turn descends to a homotopy on \(M\backslash G\). This pushes forward to the desired homotopy on \(N\).

(iv)\(\Rightarrow\)(v)
For any given \(g\), \(h\) defined as above is the desired homotopy.

(v)\(\Rightarrow\)(i)
Apply (v) to the pullback of \(k\) and push forward the resulting homotopy to \(N\).
\end{proof}
\begin{rem}
In condition \ref{enum:subgroup}, to say that \(G_{\H}\) is a sub-group of \(G\) here is via the concrete realisation (representation) of both groups acting on \(M\): the inclusion the inclusion map \(\iota : G_{\H} \to G\) is compatible with the actions of \(G_{\H}\) and \(G\) on \(M\) so that \(\iota(\varphi) \cdot x = \varphi \cdot x\) for every \(\varphi \in G_{\H}\) and \(x \in M\). Thus it is not enough for \(G_{\H}\) to be isomorphic to a subgroup of \(G\) - the isomorphism must also be compatible with the actions.
\end{rem}
The question of whether the conditions of \Cref{lem:const_neg} are satisfied are not easy to check but the lemma affords us with several possible approaches to the problem. Here we prove the following theorem giving a \emph{sufficient} condition for when \((N, k)\) admits a metric of constant, negative sectional curvature.
\begin{thm}[Integrability and constant negative sectional curvature]
\label{thm:intg_const_curv}
Let \((N, k)\) be a compact Riemannian three-manifold of strictly negative sectional curvature with the integrability condition that the tensor \(\Ob = \sqrt{\det \Ein} \Ein^{-1}\) is \emph{Codazzi}. Then \(N\) admits a metric of constant, negative sectional curvature.
\end{thm}
Here \(\Ein = \Ric - \tfrac{\Sc}{2}k\) denotes the Einstein tensor. Writing \(\Ein(X, Y) = k(\opEin(X), Y)\) we define \(\det \Ein = \det \opEin\) and \(\Ein^{-1}(X, Y) = k(\opEin^{-1} (X), Y)\) whenever \(\opEin\) is invertible. To say that \(\Ob\) is Codazzi means that the three-tensor \(\nabla \Ob\) is totally symmetric.
\begin{rem}
Note that by \Cref{lem:eins_sectional} below, in three dimensions, \(k\) has strictly negative sectional curvature if and only if \(\Ein\) is strictly positive definite. In that case, we see that \(\opEin\) is a strictly positive-definite endomorphism and in particular invertible. Thus \(\Ein\) and \(\Ob\) are well defined.
\end{rem}
The theorem is in fact a corollary of the simple \Cref{thm:intg_embed} and \cite[Theorem 1.1]{MR3344442} which says that \(N\) may be deformed to the one-sheeted hyperbolid at infinity by the Gauss curvature flow.
\begin{thm}[Integrability implies isometric embeddability]
\label{thm:intg_embed}
Let \((N, k)\) be a compact Riemannian three-manifold of strictly negative sectional curvature. Then the tensor \(\Ob = \sqrt{\det \Ein} \Ein^{-1}\) is Codazzi (integrability condition) if and only if the Riemannian universal cover \((M, g = \pi^{\ast} k)\) embeds isometrically into Minkowski space \(\R^{3,1}\) as a locally convex, co-compact, spacelike hypersurface.
\end{thm}
\begin{proof}
If \((M, g)\) embeds isometrically into \(\R^{3,1}\) then by the Codazzi-Mainardi equation \eqref{eq:codazzi}, the second fundamental form \(A\) is Codazzi. Then \Cref{lem:ein_W} below gives \(A = \Ob\), hence \(\Ob\) is Codazzi.

Conversely, suppose \(\Ob\) is Codazzi and let \(\Ob\) be such that \(\Ob(X, Y) = g(\Ob(X), Y)\). Then \(\Ob\) being Codazzi is precisely the integrability condition required to locally integrate the over-determined system
\begin{align*}
F^{\ast} \inpr{\cdot}{\cdot} &= g \\
\W(F) &= \Ob
\end{align*}
for \(F\). See for example \cite{MR1713298}[Theorem 7] or a similar argument in \cite{MR1013365}[Chapter VI.12, p. 146 and Theorem V, p.393].

Since \((M, g)\) is the universal cover of \((N, k)\) with strictly negative sectional curvature, \(M\) is diffeomorphic to \(\R^3\) by the Cartan-Hadamard theorem and we can globally integrate to obtain \(F\).
\end{proof}
\begin{rem}
Necessarily, the embedding of \((M, g)\) is as a co-compact, spacelike hypersurface (i.e. the metric, \(g\) induced by the Lorentzian metric is positive definite). Since there are no compact, spacelike hypersurfaces in Minkowksi space, it is not possible to isometrically embed \((N, k)\).
\end{rem}
\begin{proof}
[Proof of \Cref{thm:intg_const_curv}]

By \Cref{thm:intg_embed} we may embed \((M, g)\) into Minkowksi space as a locally convex, co-compact, spacelike hypersurface. By \cite[Theorem 1.1]{MR3344442}, the re-scaled Gauss curvature flow deforms \((M, g)\) smoothly to the hyperboloid at infinity with constant negative sectional curvature. Thus the flow provides a smooth homotopy from \((M, g)\) to \((M, g_{\H})\) and \Cref{lem:const_neg} gives the result since the flow preserves the isometry group of the initial metric \(g\).

That the initial isometry group is preserved follows by the geometric invariance of the flow and uniqueness of solutions: according to \cite[12. Application to the cross-curvature flow]{MR3344442}, the induced metric, \(g_t\) on \(M\) evolves by the Cross Curvature Flow introduced in \cite{MR2055396} (see also \Cref{subsec:xcf_gcf} and particularly \Cref{lem:xcf_gcf} below):
\[
\begin{cases}
\partial_t g_t &= -\adj\Ein(g_t) \\
g_0 &= g.
\end{cases}
\]
At the initial time, we have \(\varphi^{\ast} g_0 = g_0\) for every \(\varphi \in G\). Then given any \(\varphi \in G\), letting \(\bar{g}_t = \varphi^{\ast} g_t\) we have
\[
\begin{cases}
\partial_t \bar{g}_t &= \varphi^{\ast} \partial_t g_t = -\varphi^{\ast}(\adj\Ein(g_t)) = -\adj\Ein(\bar{g}_t) \\
\bar{g}_0 &= g_0 = g.
\end{cases}
\]
That is \(g_t\) and \(\bar{g}_t\) both solve the Cross Curvature Flow with the same initial condition, hence by uniqueness of solutions (\cite{MR2055396,MR2207496}, \Cref{thm:xcf_existence_uniqueness} and \Cref{subsec:xcf_existence_uniqueness} below), \(g_t = \bar{g}_t = \varphi^{\ast} g_t\) and the flow is invariant under the action of \(G\).
\end{proof}
\begin{rem}
It is an open problem as to whether the Cross Curvature Flow deforms arbitrary negatively curved metrics to the constant curvature metric. A positive result would mean the integrability assumption in \Cref{thm:intg_const_curv} could be dropped. The Cross Curvature Flow is discussed below in \Cref{sec:xcf}.
\end{rem}
\section{Geometrisation Of Three Manifolds}
\label{sec:geometrisation}

\subsection{Geometrisation of Two Manfolds}
\label{sec:geometrisation_2d}

In the case of closed two-manifolds, the situation is completely understood. In this case, there is only one sectional curvature, the Gauss curvature \(K\) equal to half the scalar curvature \(R\). By the Gauss-Bonnet theorem,
\[
\int_N K d\mu = 2\pi(1-\lambda)
\]
where \(\lambda \in \N\) is the genus. Consequently, only genus zero surfaces admit a metric of constant positive Gauss curvature, only genus one surfaces admit a metric of constant zero Gauss curvature and only genus two or greater surfaces admit a metric of constant negative curvature. In fact, the uniformisation theorem implies all closed surfaces are classified up to diffeomorphism by genus and each such surface admits a metric of constant Gauss curvature. We then immediately see that if \(M\) admits a metric of strictly negative sectional curvature, then it is hyperbolic (admits a constant negative sectional curvature metric), and likewise if \(M\) admits a metric of strictly positive sectional curvature the it is elliptic (admits a constant positive sectional curvature metric). More generally, hyperbolic surfaces are precisely those surfaces admitting a metric with negative average Gauss curvature and similarly for elliptic and parabolic (admits a constant zero sectional curvature metric).

Here then we see that the hyperbolic surfaces , with infinitely many diffeomorphism classes comprise the largest class of closed surfaces, and indeed there is only one class in positive and zero curvature respectively. These hyperbolic surfaces are precisely those presented as \(\H^2/\Gamma\) where \(\Gamma\) is a Fuschian subgroup of the isometry group \(G_{\H} = \text{PSL}(2, \R)\) of \(\H^2\). Since all compact surfaces \(M\) with metrics \(g\) of strictly negative sectional curvature admit a constant curvature metric, all such surfaces are topologically quotients \(\H^2/\Gamma\). The lifted metric \(k = \pi^{\ast} g\) is then invariant under \(\Gamma\). Thus the deck transformation group of the Riemannian cover \((\H^2, k) \to (N = \H^2/\Gamma, g)\) is precisely the Fuschian subgroup \(\Gamma\) which is a subgroup of \(G_{\H}\). That is, the constant curvature metric is invariant under the deck transformation group \(\Gamma\) as in \Cref{lem:const_neg} and hence the homotopy \(h(t) = t k + (1-t)k_{\H}\) descends to the quotient \(N = M/\Gamma\) giving a homotopy \(g(t) = t g + (1-t)g_{\H}\) from \(g\) to a constant curvature metric \(g_{\H}\).

\subsection{Thurston's Geometrisation of Three Manfolds}
\label{sec:geometrisation_3d}

The three dimensional case is more complicated than the two dimensional case, but is almost completely understood thanks to Perleman's successful completion \cite{2003math......7245P,2003math......3109P,2002math.....11159P} of Hamilton's program based on the Ricci flow \cite{Hamilton:/1982} to solve the Poincar\'e and Thurston geometrisation conjectures \cite{MR648524}. The remaining piece of the puzzle is the structure of hyperbolic, closed three manifolds. We include here a brief description and refer the reader to \cite{MR705527} and \cite{MR1435975} for in depth discussions of geometrisation and \cite{MR3186136,MR2334563,MR2460872} for expositions of the Hamilton-Perleman proof. Unless explicitly stated otherwise, the results described here may be found in these references.

The geometrisation conjecture may be stated as follows:

\begin{thm}[Thurston Geometrisation]
Every closed three manifold decomposes as a connected sum of prime manifolds, each of which may be cut along tori so that the interior of the resulting manifolds each admits a unique geometric structure of with finite volume from among a possible eight types.
\end{thm}

A prime manifold is simply a manifold that cannot be written as a non-trivial connected sum. The decomposition into prime manifolds was given by \cite{MR0142125}. To say that \(M\) admits a geometry is to say that \(M\) is covered by the geometry, and the classification into eight types of finite volume geometric structures was given by Thurston. Finally the remaining part of the theorem, that such a decomposition of prime manifolds exists was proven by Hamilton and Perleman using the Ricci flow. The eight geometries are
\[
\R^3, \S^3, \H^3, \S^2 \times \R, \H^2 \times \R, \widetilde{SL}_2(\R), \operatorname{Nil}, \operatorname{Solv}.
\]

For the non-hyperbolic geometries, in the case of \(\operatorname{Solv}\) these are precisely torus and Klein bottle bundles over \(S^1\) or the union of two twisted \(1\)-bundles over the torus or Klein bottle. The remaining six non-hyperbolic geometries are all Seifert Fibre bundles, completely determined by the Euler characteristic of the base space, \(\chi\) and the Euler number of the bundle, \(e\).

The only remaining case then is the hyperbolic case.

\section{Basic Geometry Conventions}
\label{sec:notation}
\subsection{Intrinsic Geometry Conventions}
\label{subsec:notation_intrinsic}
Given a metric \(g\), let \(\nabla = \nabla^g\) denote the Levi-Civita connection. The superscript will be dropped when the metric is clear from context. Let \(\inpr{\cdot}{\cdot}\) denote the inner-product on Minkowski space and \(D\) the corresponding Levi-Civita connection. Our conventions for the curvature tensor is
\[
\Rm(X, Y) Z = \nabla_X \nabla_Y Z - \nabla_Y \nabla_X Z - \nabla_{[X, Y]} Z
\]
and
\[
\Rm(X, Y, Z, W) = g(\Rm(X, Y) Z, W).
\]
The sectional curvature is then
\[
K(X \wedge Y) = \frac{\Rm(X, Y, Y X)}{\abs{X \wedge Y}_g^2}
\]
and \(g\) has constant sectional curvature \(K_0\) if and only if (note the minus sign!)
\[
\Rm = -\frac{K}{2} g \owedge g
\]
where
\[
\frac{1}{2} g \owedge g (X, Y, Z, W) = g(X, Z) g (Y, W) - g(X, W) g(Y, Z)
\]
denotes the Kulkarni-Nomizu product. Note that
\[
\frac{1}{2} g \owedge g (X, Y, Y, X) = -\abs{X \wedge Y}_g^2.
\]

The symmetries of \(\Rm\) imply that
\[
\Rm(X\wedge Y, Z \wedge W) = \Rm(X, Y, Z, W)
\]
is a well defined, symmetric bilinear form on \(TM \wedge TM\) inducing the (self-adjoint) curvature operator \(\opRm : TM \wedge TM \to TM \wedge TM\) determined by,
\[
g(\opRm(X\wedge Y), Z \wedge W) = \Rm(X\wedge Y, Z \wedge W).
\]
The sectional curvatures may then be expressed as
\[
K(X \wedge Y) = -\frac{\Rm(X \wedge Y, X \wedge Y)}{\abs{X \wedge Y}^2}.
\]

We define the Ricci curvature
\[
\Ric(X, Y) = \Tr \Rm(\cdot, X) Y
\]
and scalar curvature
\[
\Sc = \Tr \opRic
\]
where \(\opRic\) is the self-adjoint endomorphism defined by \(\Ric(X, Y) = g(\opRic(X), Y)\). Equivalently, \(\Sc = \Tr_g \Ric := \Tr g^{\ast} \otimes \Ric\) is the full contraction and where \(g^{\ast}\) denotes the dual metric on \(T^{\ast}M\) induced by the metric \(g\) on \(TM\).

The Einstein tensor is defined by
\[
\Ein = \Ric - \frac{\Sc}{2} g.
\]
We also have the Schouten tensor defined when \(n \geq 3\),
\[
\Sch = \frac{1}{n-2} \left(\Ric - \frac{\Sc}{2(n-1)}g \right).
\]
Obviously, given \(\Ric\) we obtain \(\Ein\) and \(\Sch\) but we may also recover \(\Ric\) from either of \(\Ein\) or \(\Sch\) so that any of the three determines the other two:
\[
\Ric = \Ein + \frac{1}{2-n} \Tr_g \Ein g
\]
and
\[
\Ric = (n-2) \Sch + \frac{n-2}{2n-3} \Tr_g \Sch g.
\]
Moreover, all three tensors are symmetric and simultaneously diagonalisable with an orthonormal basis of eigenvectors.

The Ricci decomposition of the curvature tensor can be written in terms of the three tensors as
\begin{align*}
\Rm &= -\Sch \owedge g + \Wy \\
&= -\frac{1}{n-2} \left(\Ric - \frac{R}{2(n-1)} g\right) \owedge g + \Wy \\
&= -\frac{1}{n-2} \left(\Ein  -\frac{1}{n-1} \Tr_g \Ein g\right) \owedge g + \Wy
\end{align*}
where \(\Wy = \Rm + \Sch \owedge g\) is the Weyl tensor, and the Kulkarni-Nomizu product of symmetric bilinear forms \(A, B\) is defined as
\[
\begin{split}
A \owedge B (X, Y, Z, W) &= A \owedge B (X \wedge Y, Z \wedge W) \\
&= A(X, Z) B(Y, W) + A(Y, W) B(X, Z) \\
&\quad - A(X, W) B(Y, Z) - A(Y, Z) B(W, W).
\end{split}
\]
Note the first term in the sum on the right hand side of the Ricci decomposition further decomposes into irreducible components but we don't need that here.
\subsection{Extrinsic Geometry Conventions}
\label{subsec:notation_extrinsic}
For an immersion \(F : M^n \to \R^{n,1}\) with \(M\) oriented, we define the second fundamental form by
\[
D_{F_{\ast} X} F_{\ast} Y = F_{\ast} \nabla_X Y + h(X, Y).
\]
The Gauss formula says that \(F_{\ast} \nabla_X Y = (D_{F_{\ast} X} F_{\ast} Y)^T\) is the tangential part of \(D_{F_{\ast} X} F{\ast} Y\) so that \(h(X, Y)\) is normal. The immersion is \emph{spacelike} if the induced metric \(g = F^{\ast} \inpr{\cdot}{\cdot}\) is positive definite, in which case the unit normal \(\nu\) is \emph{timelike}:
\[
\inpr{\nu}{\nu} = -1.
\]
Then we may write
\[
h(X, Y) = A(X, Y) \nu
\]
where
\[
A(X, Y) = -\inpr{D_{F_{\ast} X} F_{\ast} Y}{\nu} = \inpr{F_{\ast} Y}{D_{F_{\ast} X} \nu} = g(\W(X), Y)
\]
and \(\W = F_{\ast}^{-1} \circ D\nu \circ F_{\ast}: TM \to TM\) denotes the Weingarten map.

The basic equations of hypersurfaces (Gauss equation) in Minkowski space are
\begin{equation}
\label{eq:gauss}
\begin{split}
\Rm(X, Y) Z &= A(X, Z) \W(Y) - A(Y, Z) \W(X), \\
\Ric(X, Y) &= g(\W^2(X) - H \W(X), Y), \\
\Sc &= \|A\|^2 - H^2.
\end{split}
\end{equation}
The Codazzi-Mainardi equation is of particular importance in relation to the integrability condition in \Cref{thm:intg_const_curv}. It reads
\begin{equation}
\label{eq:codazzi}
\nabla A (X; Y, Z) := \nabla_X A (Y, Z) = \nabla_Y A(X, Z) =: \nabla A(Y; X, Z).
\end{equation}
Of course, since \(A\) is already symmetric, \(\nabla A\) is automatically symmetric in the last two slots hence the Codazzi-Mainardi equation implies \(\nabla A\) becomes fully symmetric in all three slots.

Let us round out this section with a simple lemma showing spacelike hypersurfaces, \(M\) are graphs over the \(y^{n+1} = 0\) spacelike hyperplane. This simply follows because \(M\) has a timelike normal vector field, \(\nu\) so that \(\inpr{\nu}{e_{n+1}}\) has a fixed sign. We give the proof for completeness.
\begin{lemma}
\label{lem:graph}
Let \(M^n \to \R^{n,1}\) be an immersed, spacelike hypersurface. Then \(M^n\) is embedded as a graph over the \(y^{n+1} = 0\) hyperplane.
\end{lemma}
\begin{proof}
Take any local parametrisation \(\varphi : U \subset \R^n \to \R^{n,1}\) of \(M\). For any non-zero \(X \in TU\) write
\[
d\varphi \cdot X = \bar{X} + X^{n+1} e_{n+1}
\]
with \(\bar{X} = \bar{X}^a \partial_a = \varphi^a_i X^i \partial_a \in \text{span} \{\partial_a : 1 \leq a \leq n\}\). Then since \(M\) is spacelike, we have
\[
0 < g(X, X) = \inpr{\bar{X} + X^{n+1} e_{n+1}}{\bar{X} + X^{n+1} e_{n+1}} = \|\bar{X}\|_{\R^n}^2 - (X^{n+1})^2.
\]
Letting \(\pi(y^1, \ldots, y^n, y^{n+1}) = (y^1, \ldots, y^n)\) denote the orthogonal projection onto the \(y^{n+1} = 0\) hyperplane we have \(\pi \circ \varphi (x) = (\varphi^1(x), \ldots, \varphi^n(x))\) satisfies for \(X \ne 0\),
\[
\|d(\pi \circ \varphi) \cdot X\|_{\R^n}^2 = \|\bar{X}\|_{\R^n}^2  > (X^{n+1})^2 \geq 0
\]
so that \(d(\pi \circ \varphi)\) is non-singular. By the inverse function theorem, locally \(\pi \circ \varphi\) is invertible and we find that
\[
\varphi \circ (\pi \circ \varphi)^{-1} (y^1, \ldots, y^n) = (y^1, \ldots, y^n, \varphi^{n+1} \circ (\pi \circ \varphi)^{-1} (y_1, \ldots, y_n))
\]
expresses \(M\) locally as a graph of the function \(\varphi^{n+1} \circ (\pi \circ \varphi)^{-1}\) over the \(y^{n+1} = 0\) plane. But this is true locally everywhere and since graphs are unique, \(M\) is globally a graph and hence also embedded.
\end{proof}
\subsection{The Three Dimensional Case}
\label{subsec:notation_threedim}

\subsubsection{Intrinsic geometry}
Specialising to the case \(n = 3\), we have
\[
\Sch = \Ric - \frac{R}{4} g \quad \text{and} \quad W = 0.
\]
Then the Ricci decomposition becomes
\begin{equation}
\label{eq:ricci_decomp_3d}
\Rm = -\Sch \owedge g = -\Ric \owedge g + \frac{R}{4} g \owedge g = -\Ein \owedge g + \frac{\Tr_g \Ein}{2} g \owedge g.
\end{equation}
\begin{lemma}
\label{lem:eins_sectional}
The eigenvalues, \(\lambda^{\Ein}_k\) of the Einstein tensor are the eigenvalues of the curvature operator, \(\opRm\):
\[
\opRm(E_i \wedge E_j) = \lambda^{\Ein}_k E_i \wedge E_j
,\]
where \(\{E_1, E_2, E_3\}\) is an orthonormal frame of eigenvectors for \(\Ein\) and \(i,j,k\) are distinct indices. Therefore, the \emph{principal sectional curvatures} are the negatives of the eigenvalues \(\lambda^{\Ein}_k\):
\[
K(E_i \wedge E_j) = -\lambda^{\Ein}_k
\]
and \(g\) has negative sectional curvature if and only if \(\Ein\) is positive definite.
\end{lemma}
\begin{proof}
If \(X, Y\) are orthonormal eigenvectors for \(\Ein\), then using equation \eqref{eq:ricci_decomp_3d},
\[
\begin{split}
g(\opRm(E_i \wedge E_j), E_p \wedge E_q) &= -\Ein \owedge g (E_i \wedge E_j, E_p \wedge E_q) + \frac{\Tr_g \Ein}{2} g \owedge g (E_i \wedge E_j, E_p \wedge E_q) \\
&= - \Ein(E_i, E_p) g(E_j, E_q) - g(E_i, E_p) \Ein(E_j, E_q) \\
&\quad + \Ein(E_i, E_q) g(E_j, E_p) + g(E_i, E_q) \Ein(E_j, E_p) \\
&\quad + \Tr_g \Ein g(E_i \wedge E_j, E_p \wedge E_q) \\
&= - \lambda^{\Ein}_i g(E_i, E_p) g(E_j, E_q) - \lambda^{\Ein}_j g(E_i, E_p) g(E_j, E_q) \\
&\quad + \lambda^{\Ein}_i g(E_i, E_q) g(E_j, E_p) + \lambda^{\Ein}_j g(E_i, E_q) g(E_j, E_p) \\
&\quad + \Tr_g \Ein g(E_i \wedge E_j, E_p \wedge E_q) \\
&= (\Tr_g \Ein - \lambda^{\Ein}_i - \lambda^{\Ein}_j) g(E_i \wedge E_j, E_p \wedge E_q) \\
&= g(\lambda^{\Ein}_k E_i \wedge E_j, E_p \wedge E_q).
\end{split}
\]
since \(\Tr_g \Ein = \lambda^{\Ein}_i + \lambda^{\Ein}_j + \lambda^{\Ein}_k\). Hence
\[
\opRm(E_i \wedge E_j) = \lambda^{\Ein}_k E_i \wedge E_j.
\]
The sectional curvatures then follow immediately,
\[
K (E_i \wedge E_j) = -\frac{g(\opRm(E_i \wedge E_j), E_j \wedge E_i)}{\abs{E_i \wedge E_j}^2} = -\lambda^{\Ein}_k.
\]
\end{proof}

\subsubsection{Extrinsic geometry}
The change of perspective from \(\Sch\) to \(\Ein\) is also quite useful when considering hypersurfaces in \(\R^{3,1}\). Recall the Gauss equation \eqref{eq:gauss} gave us
\[
\Ric(X, Y) = g(\W^2(X) - H \W(X), Y), \quad \Sc = \|A\|^2 - H^2.
\]
Note that
\[
\opRic (X) = \W^2(X) - H \W(X)
\]
is simultaneously diagonalisable with \(\W\) and hence so too \(\Ein\) is simultaneously diagonalisable with \(A\). We have
\begin{equation}
\label{eq:ein_extrinsic}
\Ein (X, Y) = g(\W^2(X), Y) - H A(X, Y) - \frac{\|A\|^2 - H^2}{2} g (X, Y).
\end{equation}
\begin{lemma}
\label{lem:ein_W}
We have
\[
\W = \sqrt{\det \opEin} \opEin^{-1}
\]
and hence the tensor
\[
\Ob := \sqrt{\det \opEin} \Ein^{-1} = A
\]
is Codazzi. Moreover, for distinct indices, \(i,j,k\),
\[
\lambda^{\Ein}_k = \kappa_i\kappa_j
\]
and hence
\[
\opRm (E_i \wedge E_j) = \kappa_i \kappa_j E_i \wedge E_j \Rightarrow K(E_i \wedge E_j) = -\kappa_i \kappa_j
\]
and \(g\) has negative sectional curvature if and only if \(\Ein\) is positive definite if and only if \(A\) is positive definite (that is \(M\) embeds as a locally convex hypersurface).
\end{lemma}
\begin{proof}
Denoting the eigenvalues of \(\W\) (i.e. the principal curvatures) by \(\kappa_i\), and applying equation \eqref{eq:ein_extrinsic} and \Cref{lem:eins_sectional}, we have for distinct indices \(i,j\):
\[
\begin{split}
\lambda^{\Ein}_k &= \Ein(E_k, E_k) = \kappa_k^2 - \kappa_k \sum_l \kappa_l - \frac{\sum_l \kappa_l^2 - (\sum_l \kappa_l)^2}{2} \\
&= - \kappa_k \kappa_i - \kappa_k \kappa_j + \kappa_k \kappa_i + \kappa_k \kappa_j + \kappa_i \kappa_j \\
&= \kappa_i \kappa_j.
\end{split}
\]
Thus we see that the sectional curvatures, \(K(E_i \wedge E_j) = - \lambda^{\Ein}_k = -\kappa_i\kappa_j\) are minus the product of the corresponding principal curvatures and \(\Ein\) is positive definite if and only if all the \(\kappa_i\) have the same sign. This is true either if all are negative or all are positive. By changing orientation \(\nu \mapsto -\nu\) if necessary we assume that \(\kappa_i > 0\).

Furthermore, the Gauss curvature is \(K = \det \W = \kappa_i \kappa_j \kappa_k\) and when \(A\) is positive definite, \(\W^{-1}\) is well defined, simultaneously diagonalisable with \(\W\) with eigenvalues \(\kappa_i^{-1}\) and hence
\[
\adj \W := \det \W \W^{-1} = \kappa_i \kappa_j \kappa_j \kappa_i^{-1} \delta_{il} = \kappa_j \kappa_k \delta_{il} = \lambda^{\Ein}_i \delta_{il} = \Ein.
\]
That is, the tensor \(\adj \W\) is \emph{intrinsic}. In fact, writing \(\Ein(X, Y) = g(\opEin (X), Y)\),
\[
\det \opEin = \lambda^{\Ein}_i \lambda^{\Ein}_j \lambda^{\Ein}_k = \kappa_j \kappa_k \kappa_i \kappa_k \kappa_i \kappa_j = K^2.
\]
Then since \(\opEin^{-1}\) has eigenvalues \((\lambda_i^{\Ein})^{-1} = (\kappa_j \kappa_k)^{-1}\),
\[
\sqrt{\det \opEin} \opEin^{-1} = \kappa_i \kappa_j \kappa_k (\kappa_j \kappa_k)^{-1} \delta_{il} = \kappa_i \delta_{il} = \W
\]
we see that in fact \(\W\) is itself intrinsic! Therefore, defining \(\Ein^{-1}(X, Y) = g(\opEin^{-1} (X), Y)\) we have
\[
\Ob := \sqrt{\det \opEin} \Ein^{-1} = A
\]
and hence \(\Ob\) is Codazzi since by the Codazzi-Mainardi equations, \(A\) is Codazzi.
\end{proof}
\begin{rem}
We see a strong rigidity statement that the \emph{extrinsic geometry} of embedded, spacelike hypersurfaces is completely determined by the \emph{intrinsic geometry}. Then the extrinsic condition of local convexity is equivalent to the intrinsic condition of negative sectional curvature.
\end{rem}
\section{The Cross Curvature Flow}
\label{sec:xcf}
\subsection{Definition And Basic Properties Of The Flow}
\label{subsec:xcf_defn}

Let \((N, k)\) be a closed, Riemannian manifold. The Cross Curvature Flow (in short, XCF) is the evolution equation,
\begin{equation}
\label{eq:xcf}
\begin{cases}
\partial_t k_t  &= 2 \adj\Ein(k_t), \\
k_0 &= k
\end{cases}
\end{equation}
where \(k_0\) has negative sectional curvature. When \(k_0\) has positive sectional curvature, we take instead \(\partial_t k = -2\adj\Ein\) though in this article we will not be concerned with this case.
\begin{rem}
Let \(\pi : M \to N\) be the universal cover, and \(g_t = \pi^{\ast} k_t\). Similarly to the proof of \Cref{thm:intg_const_curv}, we then have
\[
\partial_t g_t = \pi^{\ast} \partial_t k_t = \pi^{\ast} \adj\Ein(k_t) = \adj\Ein(g_t)
\]
and \(g_t\) solves the XCF with initial condition \(g_0 = \pi^{\ast} k_0\). Conversely, if \(g_t\) is a \(G\)-invariant solution of the XCF on \(M\), then there is a unique solution, \(k_t\) of the XCF on \(N\) such that \(g_t = \pi^{\ast} k_t\).
\end{rem}

The definition here makes sense in any dimension. In three dimensions, according to \Cref{lem:eins_sectional}, \(k_t\) has negative sectional curvature if and only if \(\Ein\) is positive definite. In this case we may also write
\[
\adj\Ein = \det \opEin \Ein^{-1}.
\]
In an orthonormal basis of eigenvectors \(E_1, E_2, E_3\) for \(\Ein\), with eigenvalues \(\lambda_1, \lambda_2, \lambda_3\), we have for distinct indices, \(i, k, \ell\),
\[
\adj\opEin (E_i) = \det\opEin \opEin^{-1}(E_i) = \lambda_i \lambda_k \lambda_{\ell} \frac{1}{\lambda_i} E_i = \lambda_k \lambda_{\ell} E_i
\]
where \(i,k,\ell\) are distinct indices. Thus
\[
\adj\Ein(E_i, E_j) = g(\lambda_k \lambda_{\ell} E_i, E_j) = \lambda_k \lambda_{\ell} \delta_{ij}.
\]
The tensor \(\adj\Ein\) is referred to as the \emph{cross curvature tensor}. The origin of the name is that the \(i\)'th eigenvalue of \(\adj\Ein\) is the ``cross terms'' \(\lambda_k \lambda_{\ell}\) of the remaining eigenvalues.

There is an equivalent ways to write \(\adj\Ein\) in three dimensions. In fact, both these definitions make sense in any dimension, however it is only in three dimensions that they coincide.
\begin{lemma}
\label{eq:xcf_equiv}
In three dimensions, we have
\[
\adj\Ein(X, Y) = \frac{1}{2} \Ric_{\Ein} (X, Y) := \frac{1}{2} \Tr Z \mapsto \Rm(\opEin(Z), X) Y
\]
\end{lemma}
\begin{proof}
As noted above,
\[
\adj\Ein(E_i, E_j) = \lambda_k \lambda_{\ell} \delta_{ij}.
\]
On the other hand, from \Cref{lem:eins_sectional}, we have
\[
\opRm(E_i \wedge E_j) = \lambda_k E_i \wedge E_j.
\]
Then
\[
\begin{split}
\Ric_{\Ein}(E_i, E_j) &= \Tr Z \mapsto \Rm(\opEin(Z), E_i) E_j = \sum_{m=1}^3 \Rm(\opEin(E_m), E_i, E_j, E_m) \\
&= \sum_{m=1}^3 \lambda_m g(\opRm(E_m \wedge E_i), E_m \wedge E_j) \\
&= \sum_{m=1}^3 \lambda_m \hat{\lambda}_{mi} g(E_m \wedge E_i, E_m \wedge E_j).
\end{split}
\]
where \(\hat{\lambda}_{mi} = \lambda_k\) if \(m,i,k\) are distinct indices and is zero if \(m=i\). Now, if \(i = j\), and \(i, k, \ell\) are distinct indices, the sum is over \(m=k, \ell\) giving
\[
\Ric_{\Ein}(E_i, E_j) = \left(\lambda_k \lambda_{\ell} + \lambda_{\ell} \lambda_k\right) = 2\lambda_k \lambda_{\ell}.
\]
On the other hand, if \(i \ne j\), then \(g(E_m \wedge E_i, E_m \wedge E_j) = 0\) and hence
\[
\frac{1}{2} \Ric_{\Ein}(E_i, E_j) =  \lambda_k \lambda_{\ell} \delta_{ij} = \adj\Ein(E_i, E_j)
\]
\end{proof}
\begin{rem}
In \cite[Lemma 3]{MR2055396} and \cite[Equation (3)]{MR2207496}, essentially the same result is obtained by contracting with the measure \(\mu\).
\end{rem}

For a symmetric bilinear form, let us write
\[
\nabla B (X, Y, Z) = (\nabla_X B) (Y, Z) := \nabla_X (B(Y, Z)) - B(\nabla_X Y, Z) - B(Y, \nabla_X Z).
\]
Then \(\nabla B\) is symmetric in the last two slots but not generally in all three - that is, \(B\) is not generally Codazzi. We then have two ways we may trace \(\nabla B\) using \(g\): contracting the first and second slots (or equivalently the first and third by symmetry) and contracting the second and third slots. Fixing \(X\) and then we may write the bilinear form \((\nabla B)_X (Y, Z) := \nabla B (X, Y, Z)\) as \((\nabla B)_X (Y, Z) = g((\nabla B)^{\sharp}_X (Z), Y)\) and define
\[
(\Tr_g^{23} \nabla B) (X) = \Tr (\nabla B)^{\sharp}_X.
\]
Similarly we may define \(\Tr_g^{12} \nabla B = \Tr_g^{13} \nabla B\).

We define the divergence,
\[
\dive B = \dive_g B = \Tr_g^{12} \nabla B = \Tr_g^{13} \nabla B.
\]

With this definition, it is well known that the Einstein tensor is divergence free,
\begin{equation}
\label{eq:ein_divfree}
0 = \dive_g \Ein = \Tr_g^{13} \nabla \Ein = \Tr_g^{13} \nabla \Ric - \frac{1}{2} \Tr_g^{13} \nabla (\Sc g) = \dive_g \Ric - \frac{1}{2} \nabla \Sc
\end{equation}
which is precisely the contracted second Bianchi identity.

A similar identity holds for the cross curvature tensor was given in \cite[Lemma 1]{MR2055396}. For this instead of tracing with respect to \(g\) we trace with respect to \(\adj\Ein\) (which recall by \Cref{lem:eins_sectional} is positive definite, hence a metric, when \(g\) has negative sectional curvature) so defining
\[
\dive_{\adj\Ein} B = \Tr_{\adj\Ein}^{13} \nabla B \text{ where } \nabla B(X, Y, Z) = \adj\Ein([(\nabla B)_{\adj\Ein}^{13}(Y)] (X), Z).
\]
Similarly we may define \(\Tr_{\adj\Ein}^{12}\).
\begin{lemma}
\label{lem:xcf_hamilton_integrability}
The cross curvature tensor satisfies
\[
\dive_{\adj\Ein} \adj\Ein = \frac{1}{2} \Tr^{12}_{\adj\Ein} \nabla \adj\Ein.
\]
\end{lemma}
\begin{proof}
{\color{red}Write proof}
\end{proof}
The contracted second Bianchi identity served as an integrability condition used for the original short time existence and uniqueness for the Ricci flow in \cite{Hamilton:/1982}. \Cref{lem:xcf_hamilton_integrability} serves a similar purpose for the XCF as described below in \Cref{rem:invariance_integrability}.

Before moving on, note that we may write the map \(g \mapsto \adj\Ein(g)\) as
\[
\adj\Ein(g) = \adj_g \circ \Grav_g \circ \Ric (g) = (\adj \circ \opGrav)_g (\Ric(g))
\]
where for a bilinear form \(B\) on a finite dimensional vector space with an inner product \(g\) we define the \emph{gravitation linear map}
\[
\Grav_g(B) (\cdot, \cdot) = g(\opGrav(\op{B}) \cdot, \cdot) = g\left(\left[\op{B} - \frac{1}{2} \Tr\op{B} \Id\right] \cdot, \cdot\right) = B - \frac{1}{2} \Tr_g B g
\]
where
\[
B(\cdot, \cdot) = g(\op{B} \cdot, \cdot).
\]
Similarly, the adjugate \(\adj_g B = g(\adj\op{B} \cdot, \cdot)\) is a \emph{non linear} function taking bilinear forms to bilinear forms. The construction respects compositions in the sense that
\[
(\adj \circ \Grav)_g = \adj_g \circ \Grav_g.
\]
It will be convenient to write \(\adj\Ein\) in the form,
\begin{equation}
\label{eq:adjein_op}
\adj\Ein (g) = F(g, \Ric(g)) \text{ where } F(k, B) = (\adj \circ \opGrav)_k (B).
\end{equation}
In particular, \(\adj\Ein\) depends to zero order in the \(k=g\) variable and to second order through the \(B = \Ric\) variable in \(F(k, B)\).
\subsection{Short Time Existence And Uniqueness}
\label{subsec:xcf_existence_uniqueness}

{\color{red} Probably cut this to a single paragraph}

The question of short time existence and uniqueness of solutions to geometric, parabolic equations on vector bundles raises some issues not seen in the standard theory on \(\R^n\). The most pressing issue is that the geometric invariance of the problem leads to degeneracies in the principal symbol. As a result the flow is degenerate parabolic and the standard theory must be modified to take this into account.

For the Ricci flow, \cite{Hamilton:/1982} dealt with this issue by employing the Nash-Moser inverse function theorem (see for example \cite{MR656198}) and similarly for the XCF \cite{MR2055396}. However, as pointed out by \cite{MR2207496}, whilst this method is well suited to the Ricci flow which enjoys a first order integrability condition, the integrability condition for the XCF is second order. The question of whether the technique in \cite{Hamilton:/1982} applies is certainly of interest, remaining (as far as we know) an open question in general. But there is another approach to short time existence and uniqueness that also has the advantage of simplicity in comparison with the Nash-Moser approach.

For the Ricci flow, \cite{MR697987} gave an alternative proof to short time existence and uniqueness. Then, \cite{MR2207496} adapted this proof to the XCF. The method is essentially a gauge-fixing argument whereby the diffeomorphism class is fixed in such a way to produce a new flow, referred to as the DeTurck flow, which is strictly parabolic and then standard arguments apply to give existence and uniqueness of the DeTurck flow. Existence of the original flow is then easily obtained, but there is an extra step required for uniqueness which was omitted in \cite{MR2207496}. Here we outline the complete proof.

The standard theory gives that strictly parabolic, linear equations with smooth coefficients can be uniquely solved (e.g. by a contraction mapping) for smooth solutions given initial smooth data. Then the non-linear equation will be uniquely solvable on some small time interval if the linearisation is strictly parabolic by appealing to the implicit function theorem. Writing the solution as \(g_t = \varphi_t(g_0)\), the semi-group property of the solution \(g_{t_1 + t_2} = \varphi_{t_1 + t_2} (g_0) = \varphi_{t_2} (\varphi_{t_1} (g_0))\) then ensures there is a maximal time interval \([0, T)\) (possibly with \(T = \infty\)) on which the solution is defined. We take all this as given and focus only on showing first that the XCF is not strictly parabolic, then showing strict parabolicity of the DeTurck flow, and finally proving that this leads to existence and uniqueness of the XCF.
\begin{thm}
\label{thm:xcf_existence_uniqueness}
Given any initial smooth metric \(g\) of negative sectional curvature on a closed three-manifold \(N\), there exists a unique solution \(g_t\) to the XCF,
\[
\begin{cases}
\partial_t g_t &= 2\adj\Ein(g_t) \\
g_0 &= g.
\end{cases}
\]
defined on a maximal time interval \([0, T)\) for some \(T > 0\) or \(T = \infty\).
\end{thm}
To prove the theorem, recalling the formulation \(\adj\Ein = (\adj\circ\opGrav)_g (\Ric(g))\) in equation \eqref{eq:adjein_op}, we first compute the linearisation of the operator \(\adj \circ \opGrav\). Since we will evaluate this operator at \(S = \opRic(g)\) where \(g\) has negative sectional curvature, which by \Cref{lem:eins_sectional} implies \(\Ein\) is positive definite, we may assume \(\opGrav(S) = \opEin\) is invertible, and hence \(\adj \opGrav(S) = \det(\opGrav(S)) \opGrav(S)^{-1}\).
\begin{lemma}
\label{lem:dadjG}
Let \(S : V \to V\) be an isomorphism of vector spaces. Then the linearsiation of \(\adj \circ \opGrav\) at \(S\) is given by
\[
(\adj \circ \opGrav)'_S \cdot T  = \left[\Tr(\adj\opGrav(S) \opGrav(T))\Id - \adj\opGrav(S) \opGrav(T)\right] \opGrav(S)^{-1}
\]
where
\[
(\adj \circ \opGrav)'_S \cdot T := \partial_u|_{u=0} (\adj \circ \opGrav) (S + uT).
\]
\end{lemma}
\begin{proof}
{\color{red} To write up}
\end{proof}
Recall that by \Cref{lem:eins_sectional}, \(\Ein\) is positive definite when \(g\) has negative sectional curvature, hence \(\Ein\) is itself a metric. Then for \(\xi \in T^{\ast} M\), we define,
\[
\|\xi\|_{\Ein}^2 = \Ein(\xi^{\sharp}, \xi^{\sharp})
\]
where \(\xi^{\sharp}\) is the metric raising defined by \(\xi(X) = g(\xi^{\sharp}, X)\) for any \(X \in TM\) (with the same base point of \(\xi\), \(x = \pi_{T^{\ast}M} (\xi) = \pi_{TM} (X)\)). We can also raise using \(\Ein\) and so define \(\sharp_{\Ein} \xi\) by
\[
\xi(X) = \Ein(\sharp_{\Ein} \xi, X).
\]
We can also trace bilinear forms with respect to \(\Ein\), by first defining \(\Tr_{\Ein}\) on simple bilinear forms \(\xi \otimes \eta\):
\[
\Tr_{\Ein} \xi \otimes \eta = \Ein(\xi^{\sharp}, \eta^{\sharp})
\]
and extending by bilinearity.
\begin{rem}
Apart from \(\sharp_{\Ein}\), these constructions rely also on metric raising by \(g\). In particular,
\[
\abs{\xi}^2_{\Ein} \ne \Ein(\sharp_{\Ein} (\xi), \sharp_{\Ein} (\xi))
\]
where the right hand side is what we would obtain by extending \(\Ein\) from a metric on \(TM\) to a metric on \(T^{\ast} M\) in the usual way. In particular, written in terms of indices, \(\Ein^{ij} = g^{ik}g^{\ell j} \Ein_{k\ell}\) is not the inverse matrix of \(E_{ij}\).
\end{rem}
Next we compute the principal symbol of the cross curvature tensor at a metric of negative sectional curvature. Recall that for a nonlinear, second order differential operator \(D : E \to F\) acting on vector bundles \(E, F\) over \(M\), we define
\begin{equation}
\label{eq:symbol}
\sigma_{\xi} [D_s] = \sigma_{\xi} [D'_s]
\end{equation}
where \(s, u \in \Gamma(E)\) are sections of \(E\) and
\[
D'_s (u) = \partial_w|_{w=0} D(s + w u)
\]
is the linearisation of \(D\) around \(s\). Recall also that the principal symbol, \(\sigma_{\xi} [D'_s]\) of \(D'_s\) is obtained by replacing second order derivatives by components of \(\xi\), and may be conveniently computed by
\begin{equation}
\label{eq:symbol_compute}
\sigma_{\xi} [D'_s] (u) = \lim_{w\to \infty} \left[w^{-2} e^{-w\varphi(x)} D'_s (e^{w\varphi(x)} u)(x)\right]
\end{equation}
where \(\varphi\) is any smooth function satisfying \(d\varphi_x = \xi\) and \(x = \pi_{T^{\ast}M} (\xi)\).

For a natural (e.g. invariant under change of basis) and not necessarily linear map, \(f : V \to V\) and a linear differential operator \(L\), we also have the following chain rule,
\begin{equation}
\label{eq:chain_rule}
\sigma_{\xi} [f(L))] (u) = f'_{D(u)} \circ \sigma_{\xi}[L] (u).
\end{equation}
\begin{lemma}
\label{lem:xcf_symbol}
The principal symbol of the cross curvature tensor is,
\[
\sigma_{\xi} [\adj\Ein_g] (V) = \abs{\xi}^2_{\Ein} - 2 \Sym \xi \otimes V(\sharp_{\Ein}\xi, \cdot) + \Tr_{\Ein} V \xi \otimes \xi.
\]
\end{lemma}
\begin{proof}
Making use of the chain rule \eqref{eq:chain_rule} and the formulation, \(\adj\Ein = F(g, \Ric(r)) = \adj\Ein_g(\Ric(g))\) in \eqref{eq:adjein_op} we obtain
\[
\sigma_{\xi} [\adj\Ein_g] (V) = f'_{\Ric(g)} \circ \sigma_{\xi} [\Ric_g] (V).
\]
where
\[
f(B) = \adj\Ein_g (B) = g(\adj \circ \opGrav \op{B} \cdot, \cdot).
\]
\Cref{lem:dadjG} gives that,
\[
f'_B (C) = g(\left[\Tr(\adj\opGrav(\op{B}) \opGrav(\op{C}))\Id - \adj\opGrav(\op{B}) \opGrav(\op{C})\right] \opGrav(\op{B})^{-1}
\]
According to \cite[Section 5.1]{MR2265040}, the linearisation of \(\Ric\) is
\[
\sigma_{\xi} [\Ric_g] (V) = \abs{\xi}^2 V - 2 \Sym \xi \otimes V(\xi^{\sharp}, \cdot) + \Tr_g V \xi \otimes \xi.
\]

{\color{red} Need to put it all together now}
\end{proof}
\begin{rem}
\label{rem:invariance_integrability}
We see that as with any geometric evolution equation, that the XCF is only weakly parabolic. However, the degeneracy results solely from the diffeomorphism invariance and using DeTurck's method this is easily dealt with.

Let us also now note that the Hamilton integrability condition from \Cref{lem:xcf_hamilton_integrability} and the diffeomorphism invariance are closely related. {\color{red} Describe this}
\end{rem}
\begin{proof}[Proof of \Cref{thm:xcf_existence_uniqueness}]
{\color{red} Show how to construct DeTurck flow and that it is strictly parabolic. Show how to obtain uniqueness by the analogue of harmonic map heat flow.}
\end{proof}
\subsection{Basic Equations}
\label{subsec:xcf_eq}
We start by defining \(V_{ij}=(\Ein^{-1})^{kl}g_{ik}g_{lj}\)
and
\begin{align}
u&=\log\det\Ein,\quad
\Ob_{ij}=\sqrt{\det\Ein}V_{ij},\\
\T^{kij}&=\Ein^{kl}\nabla_l \Ein^{ij},\quad \T^i=V_{jk}\T^{ijk}=\Ein^{ij}\nabla_ju.
\end{align}
For some tensor $\mathrm{L}^{ijk}$ whose traces with respect to $V_{kl}$ are zero, we have
\begin{align}
\T^{ijk}-\T^{jik}&=\mathrm{L}^{ijk}-\mathrm{L}^{jik}+\frac{1}{2}\left(\T^i\Ein^{jk}-\T^j\Ein^{ik}\right).
\end{align}
The Devil tensor is defined by
\[\Dv^{ijk}=\mathrm{L}^{ijk}-\mathrm{L}^{jik}\]
and it satisfies interesting identities such as
\begin{align}
\Dv^{ijk}=&-\Dv^{jik},\quad \Dv^{ijk}+\Dv^{kij}+\Dv^{jki}=0,\\
V_{ij}\Dv^{ijk}=&V_{ik}\Dv^{ijk}=V_{jk}\Dv^{ijk}=0,\\
V_{ij}\nabla_k\Dv^{kij}=&\frac{1}{2}|\Dv^{ijk}|_V^2.
\end{align}
\begin{lemma}
\label{lem:cubicform_codazzi}
We have the following identity.
\begin{align}
|\Dv^{ijk}|_V^2=\frac{1}{\det\Ein}|\nabla_i\Ob_{jk}-\nabla_j\Ob_{ik}|_{\Ein}^2.
\end{align}
In particular, \(\Dv\equiv 0\) if and only if \(\Ob\) is Codazzi.
\end{lemma}
\begin{proof}
Using the definition we calculate
\begin{align}
\T^{kij}=&-\Ein^{kl}\Ein^{im}\Ein^{jn}\nabla_l V_{mn}=-\Ein^{kl}\Ein^{im}\Ein^{jn}\nabla_l (e^{-\frac{1}{2}u}\Ob_{mn})\\
=&-\frac{1}{\sqrt{\det\Ein}}\Ein^{kl}\Ein^{im}\Ein^{jn}\nabla_l \Ob_{mn}+\frac{1}{2}\T^k\Ein^{ij}.
\end{align}
Thus we obtain
\begin{align*}
\T^{kij}-\T^{ikj}=&\frac{\Ein^{kl}\Ein^{im}\Ein^{jn}}{\sqrt{\det\Ein}}\left(\nabla_m \Ob_{ln}-\nabla_l \Ob_{mn}\right)+\frac{1}{2}(\T^k\Ein^{ij}-\T^i\Ein^{kj}).
\end{align*}
Hence we have the identities
\begin{align}
\Dv^{kij}&=\frac{\Ein^{kl}\Ein^{im}\Ein^{jn}}{\sqrt{\det\Ein}}\left(\nabla_m \Ob_{ln}-\nabla_l \Ob_{mn}\right)\\
|\Dv^{ijk}|_V^2\det\Ein&=|\nabla_i\Ob_{jk}-\nabla_j\Ob_{ik}|_{\Ein}^2.
\end{align}
\end{proof}
\subsection{Preserving Negative Curvature}
\label{subsec:xcf_preserving}
\begin{lemma}
The XCF preserves negative sectional curvature up to the first singular time.
\end{lemma}
\begin{proof}
{\color{red} Sketch of details referring to Hamilton/Chow computations.}
\end{proof}
\subsection{Stability Of The Hyperbolic Metric}
\label{subsec:xcf_stability}

{\color{red} Refer to \cite{MR2448593}.}

\subsection{Integrability And Embeddibility}
\label{subsec:xcf_gcf}
\begin{lemma}
\label{lem:xcf_gcf}
The induced metric under the Gauss Curvature Flow in Minkowski space evolves by XCF.
\end{lemma}
\begin{proof}
{\color{red} This just follows by \(\partial_t g = -K h\) and \Cref{lem:ein_W}}
\end{proof}
\begin{thm}
\label{thm:xcf_integrable_convergence}
Under the integrability condition, the XCF smoothly converges to a hyperbolic metric.
\end{thm}

{\color{red}For proof just apply \Cref{lem:xcf_gcf} and the non-trivial part is in \cite{MR3344442}.}

\subsection{Solitons And The Harnack Inequality}
\label{subsec:xcf_solitons}
Solitons are fixed points of the flow modulo scaling and diffeomorphism. As such, they are the expected limits (up to rescaling) of the flow. Unlike other flows such as the Ricci Flow, solitons for the XCF are very rigid.
\begin{lemma}
The only solitons to the XCF are constant curvature metrics.
\end{lemma}
\begin{proof}
By definition, a solution to the XCF is a soliton if there exists a vector field $W$ and $\lambda\in \mathbb{R}$ such that at some time
\begin{equation}\label{soliton 0}
2\lambda g_{ij}=2\sigma_{ij}+\nabla_iW_j+\nabla_jW_i,
\end{equation}
where \begin{equation}
\sigma_{ij}:=-\frac{1}{2}\Ein^{kl}\Rm_{ikjl}.
\end{equation}
For an expanding soliton, (\ref{soliton 0}) holds with $\lambda=\frac{1}{4t}.$ By straightforward calculations
\begin{align}\label{trace}
\lambda \Tr_g \Ein=3\det\Ein+\Ein^{ij}\nabla_iW_j.
\end{align}
and
\begin{align}\label{equ00}
\lambda \nabla_l \Tr_g \Ein=3\nabla_l\det\Ein+\nabla_l \Ein^{ij}\nabla_iW_j+\Ein^{ij}\nabla^2_{i,l}W_j+2\sigma_l^mW_m.
\end{align}
Now using
\begin{align}\label{eq: soliton derv}
2\nabla_l\sigma_{ij}+\nabla^2_{l,i}W_j+\nabla^2_{l,j}W_i=0
\end{align}
we derive
\begin{align}
\lambda \nabla_l \Tr_g \Ein&=3\nabla_l\det\Ein+(\lambda g_{ij}-\sigma_{ij})\nabla_l \Ein^{ij}+\Ein^{ij}\nabla^2_{i,l}W_j+2\sigma_l^mW_m\nonumber\\
&=3\nabla_l\det\Ein+\lambda \nabla_l \Tr_g \Ein-\sigma_{ij}\nabla_l \Ein^{ij}+\Ein^{ij}(-2\nabla_i\sigma_{lj}-\nabla^2_{i,j}W_l)+2\sigma_l^mW_m.
\end{align}
Dividing both sides by $\det\Ein$ yields
\begin{align}
3\nabla_lu-V_{ij}\nabla_l \Ein^{ij}-\frac{\Ein^{ij}}{\det\Ein}\nabla_l\sigma_{ij}-\frac{1}{\det\Ein}(\Box W_l-2\sigma_l^kW_k)=0,
\end{align}
where \(\Box:=\Ein^{ij}\nabla^2_{i,j}.\)
Now in view of the identities
\begin{equation}\label{eq0}
\frac{\Ein^{ij}}{\det\Ein}\nabla_l\sigma_{ij}=2\nabla_lu,~\mbox{and}~ V_{ij}\nabla_l\Ein^{ij}=\nabla_lu
\end{equation}
we arrive at
\begin{align}
\Box W_l-2\sigma_l^kW_k=0\Rightarrow W^l\Box W_l-2\sigma_l^kW_kW^l=0.
\end{align}
The second identity reads
\begin{align}\label{soliton equation for W}
\Box \frac{1}{2}|W|^2-\Ein^{kl}g^{ij}\nabla_kW_i\nabla_lW_j-2\sigma^{kl}W_kW_l=0.
\end{align}
By \cite{MR2055396}[Lemma 1], $\nabla_i\Ein^{ij}=0.$ Therefore, $\int \Box fd\mu=0$ for any smooth function $f$ on $M$.
Now integrating (\ref{soliton equation for W}) against $d\mu$ and taking into account that $\Ein_{ij}$ and $\sigma_{ij}$ are both positive definite proves that $W\equiv0$. Hence, by (\ref{soliton 0}),
\[(\det\Ein)^2=\frac{\det \sigma_{ij}}{\det g_{ij}}=\lambda ^3\Rightarrow \det\Ein=\lambda^{\frac{3}{2}}.\]
Moreover, in view of (\ref{trace}), we have $3\det\Ein=\lambda \Tr_g \Ein$ and finally \[3(\det\Ein)^{\frac{1}{3}}=\Tr_g \Ein.\] Thus the metric has constant negative curvature.
\end{proof}
There is a Harnack inequality for integrable solutions of the XCF.
\begin{thm}
\label{thm:harnack}
The following Li-Yau-Hamilton Harnack inequality holds for the XCF,
\[
\partial_t \sqrt{\det\Ein}-\frac{1}{\sqrt{\det\Ein}}\Ein^{ij}\nabla_i\sqrt{\det\Ein}\nabla_j\sqrt{\det\Ein}+\frac{3}{4t}\sqrt{\det\Ein}\geq0.
\]
\end{thm}
\begin{proof}
{\color{red}Use embedding and BIS4.}
\end{proof}
\subsection{Monotonicity}
\label{subsec:xcf_volume}

\begin{thm}
\label{thm:volume_monotonicity}
Under the XCF of negative sectional curvature,
\[
I(M_t):=\int \sqrt{\det\Ein}d\mu
\]
is non-decreasing. Moreover, \(\frac{d}{dt} I(M_t)= 0\) if and only if \(\Ob\) is Codazzi if and only if the Riemannian universal cover \((M, g = \pi^{\ast} k)\) embeds isometrically into Minkowski space \(\R^{3,1}\) as a locally convex, co-compact, spacelike hypersurface.
\end{thm}
\begin{proof}
By \cite[Proposition 9]{MR2055396}, we have
\begin{align}
(\partial_t-\Box)(\det\Ein)^\eta=(\frac{\eta}{2}|\Dv^{ijk}|_V^2+\frac{\eta}{2}(1-2\eta)|\T^i|^2_V-2\eta\mathrm{H})(\det\Ein)^\eta,
\end{align}
where $\mathrm{H}=\operatorname{Tr}_g\sigma.$ Therefore, we have the identity
\begin{align}
\frac{d}{dt}\int(\det\Ein)^\eta d\mu=\int (\frac{\eta}{2}|\Dv^{ijk}|_V^2+\frac{\eta}{2}(1-2\eta)|\T^i|^2_V+(1-2\eta)\mathrm{H})(\det\Ein)^\eta d\mu.
\end{align}
In particular, for $\eta=\frac{1}{2}$ we obtain
\begin{align}
\frac{d}{dt}I(M_t)=\frac{1}{4}\int |\Dv^{ijk}|_V^2\sqrt{\det\Ein}d\mu.
\end{align}
Now apply \Cref{lem:cubicform_codazzi} and \Cref{thm:intg_embed}.
\end{proof}
\begin{rem}
Characterisation of solutions to the XCF where \(\frac{d}{dt} I(M_t) = 0\) was posed as a question in \cite{MR2055396}, page 9. The previous theorem provides a complete answer.
\end{rem}
There is another important monotone quantity along the XCF that measures the difference of the metric from hyperbolic.
\begin{thm}\label{thm:hyperboliicity}
Under the XCF
\[J(M_t):=\int \frac{\operatorname{Tr}_g\Ein}{3}- (\det\Ein)^{\frac{1}{3}}d\mu\]
is non-increasing. Moreover, \(\frac{d}{dt}J(M_t) = 0\) if and only if $g$ has constant curvature.
\end{thm}
\begin{proof}
See Theorem 8 of \cite{MR2055396}.
\end{proof}

\subsection{Special Solutions}
\label{subsec:xcf_special}


{\color{red} List results for homogeneous spaces here and also the torus result, particularly as relating to Thurston.}

\subsection{The General Case}
\label{subsec:xcf_general}

{\color{red} Some words about the devil tensor and the difficulties it causes. Perhaps give the evolution of the cubic form?}

\section*{}
% Empty section for bib so I don't lose it when collapsing sections

\printbibliography

\end{document}
