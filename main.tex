\documentclass[a4paper,12pt]{amsart}

%\usepackage{etoolbox}
%\makeatletter
%\let\ams@starttoc\@starttoc
%\makeatother
%\makeatletter
%\let\@starttoc\ams@starttoc
%\patchcmd{\@starttoc}{\makeatletter}{\makeatletter\parskip\z@}{}{}
%\makeatother

%\usepackage[parfill]{parskip}
\usepackage{vmargin}
\usepackage[colorlinks=true,linkcolor=blue,citecolor=blue,urlcolor=blue]{hyperref}
\usepackage{bookmark}
\usepackage{amsthm,thmtools,amssymb,amsmath,amscd,amsfonts}
\usepackage{mathrsfs}
\usepackage{stmaryrd}


\usepackage[bibstyle=authoryear,citestyle=authoryear,backend=bibtex]{biblatex}
\bibliography{Bibliography}

\usepackage{fancyhdr}
\usepackage{esint}

\usepackage{enumerate}

\usepackage{pictexwd,dcpic}

\usepackage{graphicx}
\usepackage[utf8]{inputenc}

\declaretheorem[name=Theorem,numberwithin=section]{thm}
\declaretheorem[name=Remark,style=remark,sibling=thm]{rem}
\declaretheorem[name=Lemma,sibling=thm]{lemma}
\declaretheorem[name=Proposition,sibling=thm]{prop}
\declaretheorem[name=Definition,style=definition,sibling=thm]{defn}
\declaretheorem[name=Corollary,sibling=thm]{cor}
\declaretheorem[name=Assumption,style=remark,sibling=thm]{ass}
\declaretheorem[name=Example,style=remark,sibling=thm]{example}


\numberwithin{equation}{section}

\usepackage{cleveref}
\crefname{lemma}{Lemma}{Lemmata}
\crefname{prop}{Proposition}{Propositions}
\crefname{thm}{Theorem}{Theorems}
\crefname{cor}{Corollary}{Corollaries}
\crefname{defn}{Definition}{Definitions}
\crefname{example}{Example}{Examples}
\crefname{rem}{Remark}{Remarks}
\crefname{ass}{Assumption}{Assumptions}
\crefname{not}{Notation}{Notation}
\crefname{section}{Section}{Sections}

%Symbols
\renewcommand{\~}{\tilde}
\renewcommand{\-}{\bar}
\newcommand{\bs}{\backslash}
\newcommand{\cn}{\colon}
\newcommand{\sub}{\subset}

\newcommand{\N}{\mathbb{N}}
\newcommand{\R}{\mathbb{R}}
\newcommand{\Z}{\mathbb{Z}}
\renewcommand{\S}{\mathbb{S}}
\renewcommand{\H}{\mathbb{H}}
\newcommand{\C}{\mathbb{C}}
\newcommand{\K}{\mathbb{K}}
\newcommand{\Di}{\mathbb{D}}
\newcommand{\B}{\mathbb{B}}
\newcommand{\8}{\infty}

%Greek letters
\renewcommand{\a}{\alpha}
\renewcommand{\b}{\beta}
\newcommand{\g}{\gamma}
\renewcommand{\d}{\delta}
\newcommand{\e}{\epsilon}
\renewcommand{\k}{\kappa}
\renewcommand{\l}{\lambda}
\renewcommand{\o}{\omega}
\renewcommand{\t}{\theta}
\newcommand{\s}{\sigma}
\newcommand{\p}{\varphi}
\newcommand{\z}{\zeta}
\newcommand{\vt}{\vartheta}
\renewcommand{\O}{\Omega}
\newcommand{\D}{\Delta}
\newcommand{\G}{\Gamma}
\newcommand{\T}{\Theta}
\renewcommand{\L}{\Lambda}

%Mathcal Letters
\newcommand{\cL}{\mathcal{L}}
\newcommand{\cT}{\mathcal{T}}
\newcommand{\cA}{\mathcal{A}}
\newcommand{\cW}{\mathcal{W}}

%Mathematical operators
\newcommand{\INT}{\int_{\O}}
\newcommand{\DINT}{\int_{\d\O}}
\newcommand{\Int}{\int_{-\infty}^{\infty}}
\newcommand{\del}{\partial}

\newcommand{\inpr}[2]{\left\langle #1,#2 \right\rangle}
\newcommand{\abs}[1]{\left\lvert{#1}\right\rvert}
\newcommand{\fr}[2]{\frac{#1}{#2}}
\newcommand{\x}{\times}
\DeclareMathOperator{\Tr}{Tr}
\DeclareMathOperator{\Id}{Id}

\DeclareMathOperator{\dive}{div}
\DeclareMathOperator{\id}{id}
\DeclareMathOperator{\pr}{pr}
\DeclareMathOperator{\Diff}{Diff}
\DeclareMathOperator{\supp}{supp}
\DeclareMathOperator{\graph}{graph}
\DeclareMathOperator{\osc}{osc}
\DeclareMathOperator{\const}{const}
\DeclareMathOperator{\dist}{dist}
\DeclareMathOperator{\loc}{loc}
\DeclareMathOperator{\grad}{grad}
\DeclareMathOperator{\Ric}{Ric}
\DeclareMathOperator{\opRic}{\mathcal{R}ic}
\DeclareMathOperator{\Rm}{Rm}
\DeclareMathOperator{\Sc}{R}
\DeclareMathOperator{\Ein}{G}
\DeclareMathOperator{\opEin}{\mathcal{G}}
\DeclareMathOperator{\Sch}{P}
\DeclareMathOperator{\W}{\mathcal{W}}
\DeclareMathOperator{\inj}{inj}
\DeclareMathOperator{\adj}{adj}
\DeclareMathOperator{\Sym}{Sym}

%Environments
\newcommand{\Theo}[3]{\begin{#1}\label{#2} #3 \end{#1}}
\newcommand{\pf}[1]{\begin{proof} #1 \end{proof}}
\newcommand{\eq}[1]{\begin{equation}\begin{alignedat}{2} #1 \end{alignedat}\end{equation}}
\newcommand{\IntEq}[4]{#1&#2#3	 &\quad &\text{in}~#4,}
\newcommand{\BEq}[4]{#1&#2#3	 &\quad &\text{on}~#4}
\newcommand{\br}[1]{\left(#1\right)}

%Logical symbols
\newcommand{\Ra}{\Rightarrow}
\newcommand{\ra}{\rightarrow}
\newcommand{\hra}{\hookrightarrow}
\newcommand{\mt}{\mapsto}

%Names
\newcommand{\holder}{H\"older}

%Fonts
\newcommand{\mc}{\mathcal}
\renewcommand{\it}{\textit}
\newcommand{\mrm}{\mathrm}

%Spacing
\newcommand{\hp}{\hphantom}


%\parindent 0 pt

\protected\def\ignorethis#1\endignorethis{}
\let\endignorethis\relax
\def\TOCstop{\addtocontents{toc}{\ignorethis}}
\def\TOCstart{\addtocontents{toc}{\endignorethis}}


\newcommand{\note}[1]{\Rd {\bf[[ #1 ]]} \Bk}

\author{Paul Bryan and Mohammad N. Ivaki and Julian Scheuer}


%\author[P. Bryan]{Paul Bryan}
%\address{Department of Mathematics, Macquarie University NSW 2109, Australia}
%\email{\href{mailto:paul.bryan@uq.edu.au}{paul.bryan@uq.edu.au}}
%\urladdr{\href{http://pabryan.github.io}{http://pabryan.github.io/}}

%\author[M.N. Ivaki]{Mohammad N. Ivaki}
%\address{Department of Mathematics, University of Toronto, Ontario, M5S 2E4, Canada}
%\email{\href{mailto:m.ivaki@utoronto.ca}{m.ivaki@utoronto.ca}}

%\author[J. Scheuer]{Julian Scheuer}
%\address{Department of Mathematics, Columbia University New York, NY 10027, USA}
%\email{\href{mailto:jss2291@columbia.edu}{jss2291@columbia.edu}}
%\urladdr{\href{https://home.mathematik.uni-freiburg.de/scheuer/}{https://home.mathematik.uni-freiburg.de/scheuer/}}

\DeclareMathOperator{\Ob}{O}
\DeclareMathOperator{\opOb}{\mathcal{O}}
\DeclareMathOperator{\T}{T}
\DeclareMathOperator{\Dv}{D}
\DeclareMathOperator{\xcf}{\sigma}
\DeclareMathOperator{\dtxcf}{\xcf_{\operatorname{DT}}}
\renewcommand{\L}{\ensuremath{\operatorname{L}}}
\DeclareMathOperator{\dtrf}{\Ric_{\operatorname{DT}}}


\begin{document}

\title[Negatively Curved Three Manifolds]{Negatively Curved Three-Manifolds, Hyperbolic Metrics, Isometric Embedding In Minkowski Space And The Cross Curvature Flow}

\date{\today}


\subjclass[2010]{58J35, 35K10, 58B20}
\keywords{Negative curvature, embedding, Minkowski, space-like}

\begin{abstract}
This short note is a mostly expository article examining negatively curved three-manifolds. We look at some rigidity properties related to isometric embeddings into Minkowski space and the use of the Cross Curvature Flow (XCF) to study the space of negatively curved metrics on three-manifolds. For the XCF, we examine the relationship between integrability and embedability showing that solutions with fixed Einstein volume are precisely the integrable solutions, answering a question posed by Chow and Hamilton when they introduced the XCF. We describe the differential Li-Yau-Hamilton Harnack inequality for integrable solutions and show that solitons have constant curvature.
\end{abstract}

\maketitle

\section{Embeddability and hyperbolic metrics}
\label{sec:intro}

Let \((M, g)\) be a compact, Riemannian manifold with strictly negative curvature. Let \(\pi\colon (\tilde{M}, \tilde{g}) \to (M, g)\) be the Riemannian universal cover so that \(\pi : \tilde{M} \to M\) is a covering map with \(\tilde{M}\) simply connected and \(\tilde{g} = \pi^{\ast} g\). Let \(G\) denote the deck transformation group of the cover and observe that \(\tilde{g}\) is invariant under \(G\). That is, \(G \leq \text{Diff}(\tilde{M})\) is a group of diffeomorphisms of \(\tilde{M}\) and \(\varphi^{\ast} \tilde{g} = \tilde{g}\) for all \(\varphi \in G\) so that \(G\) acts by isometry on \((\tilde{M}, \tilde{g})\). Then \(\tilde{g}\) induces a metric \(\bar{g}\) on the quotient \(\tilde{M}/G\) such that
\[
(\tilde{M}/G, \bar{g}) \underset{\simeq}{\to} (M, g)
\]
is an isometry and the quotient map \(\tilde{M} \to \tilde{M}/G\) is just \(\pi\) under this identification. Then \((\tilde{M}/G, \bar{g})\) is a compact Riemannian quotient and we say \((\tilde{M}, \tilde{g})\) is a co-compact Riemannian manifold.

Now, since \((M, g)\) has strictly negative sectional curvature, so does \((\tilde{M}, \tilde{g})\) hence by the Cartan-Hadamard theorem, \(\tilde{M} \simeq \R^3\) is diffeomorphic to \(\R^3\) via the exponential map. In particular we may equip \(\tilde{M}\) with the hyperbolic metric, \(\tilde{g}_{\H}\) of constant, negative sectional curvature equal to \(-1\). Let us write \(G_{\H}\) for the isometry group of \((\tilde{M}, \tilde{g}_{\H})\).

On \(\tilde{M}\), there is a simple, smooth homotopy from \(\tilde{g}\) to \(\tilde{g}_{\H}\):
\[
\tilde{h}(t) = t\tilde{g} + (1-t)\tilde{g}_{\H}, \quad t \in [0, 1].
\]
This gives rise to the following simple lemma:

\begin{lemma}
\label{lem:const_neg}

Let \((M, g)\) be a compact manifold of strictly negative sectional curvature. Then the following statements are equivalent:
\begin{enumerate}[(i)]
\item \label{enum:neg_met} \(M\) admits a metric of constant, negative sectional curvature.
\item \label{enum:deck_met} \(\tilde{g}_{\H}\) is invariant under \(G\).
\item \label{enum:subgroup} G is a subgroup of \(G_{\H}\).
\item \label{enum:homo_met} \(g\) is smoothly homotopic to a metric of constant, negative sectional curvature.
\item \label{enum:homo_deck} Every \(G\)-invariant metric \(\tilde{g}\) on \(\tilde{M}\) is smoothly homotopic to \(\tilde{g}_{\H}\) via a smooth \(G\)-invariant homotopy.
\end{enumerate}
\end{lemma}

\begin{proof}
(i)\(\ \Rightarrow\ \)(ii) 
The pullback of a constant curvature metric under the Riemannian covering $\pi$ is the hyperbolic one, which is thus \(G\)-invariant.

(ii)\(\ \Rightarrow\ \)(iii)
Clear, since \(G_{\H}\) is the whole isometry group.

(iii)\(\ \Rightarrow\ \)(iv)
Since \(\tilde{g}\) and \(\tilde{g}_{\H}\) are invariant under G, so is \(\tilde{h}(t)\), which in turn descends to a homotopy on \(\tilde{M}/G\). This pushes forward to the desired homotopy, \(h\) on \(M\).

(iv)\(\ \Rightarrow\ \)(v)
For any given \(\tilde{g}\), the push forward \(h\) of \(\tilde{h}\) defined above is the desired homotopy.

(v)\(\ \Rightarrow\ \)(i)
Apply (v) to the pullback of \(g\) and push forward the resulting homotopy to \(M\).
\end{proof}

The question of whether the conditions of \Cref{lem:const_neg} are satisfied are not easy to check but the lemma affords us with several possible approaches to the problem. In this section we prove  \Cref{thm:intg_const_curv}, which gives a \emph{sufficient} condition for when \((M, g)\) admits a metric of constant, negative sectional curvature.

Before we can state it, let us agree on some notation and conventions. Given a metric \(g\) with Levi-Civita connection $\nabla$ on a manifold $M$, our conventions for the curvature tensor are
\[
\begin{split}
\Rm(X, Y) Z &= \nabla_X \nabla_Y Z - \nabla_Y \nabla_X Z - \nabla_{[X, Y]} Z, \\
\Rm(X, Y, Z, W) &= g(\Rm(X, Y) Z, W).
\end{split}
\]
Then the Ricci and scalar curvature are defined by
\eq{
\Ric(X, Y)&= \Tr \Rm(\cdot, X) Y,\\
		\Sc &=\Tr_{g}\Ric,
}
where $\Tr$ is the trace of an endomorphism and $\Tr_{g}$ is the trace of a bilinear form with respect to the metric $g$.

We define the Einstein tensor by
\eq{\Ein=\Ric-\frac{\Sc}{2}g.} 

We may write the Ricci decomposition of the curvature tensor in three dimensions in the form
\[
\Rm = -\Ein \owedge g + \frac{\Tr_g \Ein}{2} g \owedge g
\]
where \(\owedge\) denotes the Kulkarni-Nomizu product. The sectional curvatures are
\[
K(X \wedge Y) = \frac{\Rm(X, Y, Y, X)}{\abs{X \wedge Y}^2}.
\]
The curvature operator \(\opRm\) is defined by
\[
\Rm(X, Y, Z, W) = \Rm(X \wedge Y, W \wedge Z) = g(\opRm(X \wedge Y), W \wedge Z).
\]

Then from the Ricci decomposition, given an orthonormal basis of eigenvectors \(E_i\) for \(\Ein\) with eigenvalues \(\lambda_i\) we have
\[
\begin{split}
g(\opRm(E_i \wedge E_j), E_p \wedge E_q) &= -\Ein \owedge g (E_i \wedge E_j, E_p \wedge E_q) + \frac{\Tr_g \Ein}{2} g \owedge g (E_i \wedge E_j, E_p \wedge E_q) \\
&= - \Ein(E_i, E_p) g(E_j, E_q) - g(E_i, E_p) \Ein(E_j, E_q) \\
&\quad + \Ein(E_i, E_q) g(E_j, E_p) + g(E_i, E_q) \Ein(E_j, E_p) \\
&\quad + \Tr_g \Ein g(E_i \wedge E_j, E_p \wedge E_q) \\
&= - \lambda_i g(E_i, E_p) g(E_j, E_q) - \lambda_j g(E_i, E_p) g(E_j, E_q) \\
&\quad + \lambda_i g(E_i, E_q) g(E_j, E_p) + \lambda_j g(E_i, E_q) g(E_j, E_p) \\
&\quad + \Tr_g \Ein g(E_i \wedge E_j, E_p \wedge E_q) \\
&= (\Tr_g \Ein - \lambda_i - \lambda_j) g(E_i \wedge E_j, E_p \wedge E_q) \\
&= g(\lambda_k E_i \wedge E_j, E_p \wedge E_q).
\end{split}
\]
Thus
\[
\opRm(E_i \wedge E_j) = \lambda_k E_i \wedge E_j
\]
and the eigenvalues of \(\opRm\) are precisely the eigenvalues of \(\Ein\). The sectional curvatures are then
\[
K (E_i \wedge E_j) = -\frac{g(\opRm(E_i \wedge E_j), E_j \wedge E_i)}{\abs{E_i \wedge E_j}^2} = -\lambda_k.
\]

Therefore \(\Ein\) is positive definite (respectively negatively definite) if and only if the sectional curvatures are negative (respectively positive). In the case of negative sectional curvature, \(\Ein\) is hence a metric. Writing \(\Ein(X, Y) = g(\opEin(X), Y)\), define \(\det \Ein = \det \opEin\) and \(\Ein^{-1}(X, Y) = g(\opEin^{-1} (X), Y)\).

Now we have the following theorem. We say that a symmetric \((0,2)\)-tensor \(T\) is \emph{Codazzi} if the covariant three-tensor \(\nabla T\) is totally symmetric.

\begin{thm}[Integrability and constant negative sectional curvature, {\cite[Section 11]{MR3344442}}]
\label{thm:intg_const_curv}

Let \((M, g)\) be a compact Riemannian three-manifold of strictly negative sectional curvature with the integrability condition that the tensor \(\Ob = \sqrt{\det \Ein} \Ein^{-1}\) is \emph{Codazzi}. Then \(M\) admits a metric of constant, negative sectional curvature.
\end{thm}

\Cref{thm:intg_const_curv} follows from the embeddability \Cref{thm:intg_embed} and \cite[Theorem 1.1]{MR3344442}, which says that \(N\) may be deformed to the one-sheeted hyperboloid at infinity by the Gauss curvature flow.
Before we can state and prove \Cref{thm:intg_embed}, we need some more notation concerning extrinsic geometry.

Let \(\inpr{\cdot}{\cdot}\) denote the inner-product on Minkowski space and \(D\) the corresponding Levi-Civita connection. For a spacelike immersion \(F\cn M^n \to \R^{n,1}\) with \(M\) oriented, we define the second fundamental form $A$ with respect to a timelike, unit normal field $\nu$ by
\[
D_{F_{\ast} X} F_{\ast} Y = F_{\ast} \nabla_X Y + A(X,Y)\nu.
\]
We also define the Weingarten map via
\[
A(X, Y) =  g(\W(X), Y)
\] 
and write $H = \Tr_{g}A = \Tr \W$.

The basic equations of hypersurfaces (Gauss equation) in Minkowski space are
\begin{equation}
\label{eq:gauss}
\begin{split}
\Rm(X, Y) Z &= A(X, Z) \W(Y) - A(Y, Z) \W(X), \\
\Ric(X, Y) &= g(\W^2(X) - H \W(X), Y), \\
\Sc &= \|A\|^2 - H^2.
\end{split}
\end{equation}
We can also relate the eigenvalues $\lambda_{i}$ of $\opEin$ with the principal curvatures $\kappa_{i}$ of the embedding. Namely, for distinct indices \(i,j,k\) we calculate with the help of \eqref{eq:gauss},
\eq{
\lambda_k & = \kappa_k^2 - \kappa_k \sum_l \kappa_l - \frac{\sum_l \kappa_l^2 - (\sum_l \kappa_l)^2}{2} \\
&= - \kappa_k \kappa_i - \kappa_k \kappa_j + \kappa_k \kappa_i + \kappa_k \kappa_j + \kappa_i \kappa_j \\
&= \kappa_i \kappa_j.
}

Hence there holds
\eq{\label{lem:ein_W}
\W = \sqrt{\det \opEin} \opEin^{-1},
}
since these endomorphisms are simultaneously diagonalizable and share the same eigenvalues.

Therefore, \(\Ein > 0\) if and only if all the principal curvatures \(\kappa_i\) have the same sign (which may take to be positive by swapping \(\nu\) with \(-\nu\) if necessary). That is, \(g\) has negative sectional curvature if and only if \(\Ein > 0\) if and only if \(F(\tilde{M})\) is a locally convex, co-compact, spacelike hypersurface.

\begin{rem}
We see a strong rigidity statement that the \emph{extrinsic geometry} of embedded, spacelike hypersurfaces is completely determined by the \emph{intrinsic geometry}. The extrinsic condition of local convexity is equivalent to the intrinsic condition of negative sectional curvature.
\end{rem}

Now we may give an intrinsic characterisation of when \((\tilde{M}, \tilde{g})\) embeds isometrically into Minkowski space as precisely when \(\Ob\) is Codazzi.

\begin{thm}[Integrability implies isometric embeddability]
\label{thm:intg_embed}

Let \((M, g)\) be a compact Riemannian three-manifold of strictly negative sectional curvature. Then the tensor \(\Ob = \sqrt{\det \Ein} \Ein^{-1}\) is Codazzi if and only if the Riemannian universal cover \((\tilde{M}, \tilde{g})\) embeds isometrically into Minkowski space \(\R^{3,1}\) as a locally convex, co-compact, spacelike hypersurface.
\end{thm}

\begin{proof}
First suppose \((\tilde{M}, \tilde{g})\) embeds isometrically into \(\R^{3,1}\). Since Minkowski space is flat, the second fundamental form \(A\) is Codazzi. Then \Cref{lem:ein_W} gives \(A = \Ob\), hence \(\Ob\) is Codazzi.

Conversely, suppose \(\Ob\) is Codazzi. Since $\Ob$ solves the contracted Gauss equation \eqref{eq:gauss}, this is precisely the integrability condition required to locally integrate the over-determined system
\begin{align*}
F^{\ast} \inpr{\cdot}{\cdot} &= g \\
A(F) &= \Ob
\end{align*}
for \(F\). See for example \cite{MR1713298}[Theorem 7] for a similar argument in \cite{MR1013365}[Chapter VI.12, p. 146 and Theorem V, p.393].

Since \((\tilde{M}, \tilde{g})\) is the universal cover of \((M, g)\) with strictly negative sectional curvature, \(\tilde{M}\) is diffeomorphic to \(\R^3\) by the Cartan-Hadamard theorem and we can globally integrate to obtain \(F\).
\end{proof}

\begin{proof}
[Proof of \Cref{thm:intg_const_curv}]

By \Cref{thm:intg_embed} we may embed \((\tilde{M}, \tilde{g})\) into Minkowski space as a locally convex, co-compact, spacelike hypersurface. By \cite[Theorem 1.1]{MR3344442}, the rescaled Gauss curvature flow deforms \((\tilde{M}, \tilde{g})\) smoothly to the hyperboloid at infinity with constant negative sectional curvature. Thus the flow provides a smooth homotopy from \((\tilde{M}, \tilde{g})\) to \((\tilde{M}, \tilde{g}_{\H})\).

According to \cite[11. Application to the cross-curvature flow]{MR3344442} (see also \Cref{lem:xcf_gcf}), the induced metric \(\tilde{g}_t\) on \(\tilde{M}\) evolves by the Cross Curvature Flow introduced in \cite{MR2055396} (see also \Cref{lem:xcf_gcf} below):
\[
\begin{cases}
\partial_t \tilde{g}_t &= 2\det E(\tilde{g}_{t}) E^{-1}(\tilde{g}_{t}) \\
\tilde{g}_0 &= \tilde{g}.
\end{cases}
\]
At the initial time, we have \(\varphi^{\ast} \tilde{g}_0 = \tilde{g_0}\) for every \(\varphi \in G\). Then given any \(\varphi \in G\), \(\bar{g}_t = \varphi^{\ast} \tilde{g}_t\) is also a solution to the Cross Curvature Flow with the same initial condition. Hence by uniqueness of solutions (\cite{MR2055396,MR2207496}, \Cref{thm:xcf_existence_uniqueness} and \Cref{subsec:xcf_existence_uniqueness} below), \(\tilde{g}_t = \bar{g}_t = \varphi^{\ast} \tilde{g}_t\) and the flow is invariant under the action of \(G\). \Cref{lem:const_neg} then gives the result.
\end{proof}

The following conjecture suggests the integrability assumption in \Cref{thm:intg_const_curv} could be dropped.

\begin{conj}[\cite{MR2055396}]
\label{conj:chow_hamilton}

The XCF deforms arbitrary negatively curved metrics to a hyperbolic metric.
\end{conj}

Evidence for this conjecture includes convergence in the integrable case from \Cref{thm:intg_const_curv}, asymptotic stability of the hyperbolic metric under XCF (\cite{MR2448593}, \Cref{thm:hyperbolic_stability}, monotonicity of an integral quantity measuring the deviation from constant curvature (\cite{MR2055396}, \Cref{thm:hyperbolicity}) as well as some results for other special cases like long time existence for the \(2\pi\)-metric of Gromov and Thurston on solid tori, and locally homogeneous solutions \cite{}.

\section{Geometrisation Of Three Manifolds}
\label{sec:geometrisation}

\subsection{Geometrisation of Two Manifolds}
\label{sec:geometrisation_2d}

In the case of closed two-manifolds, the situation is completely understood. In this case, there is only one sectional curvature, the Gauss curvature \(K\) equal to half the scalar curvature \(R\). By the Gauss-Bonnet theorem,
\[
\int_M K d\mu = 2\pi(1-\lambda)
\]
where \(\lambda \in \N\) is the genus. Consequently, only genus zero surfaces admit a metric of constant positive Gauss curvature, only genus one surfaces admit a metric of constant zero Gauss curvature and only genus two or greater surfaces admit a metric of constant negative curvature. In fact, the uniformisation theorem implies all closed surfaces are classified up to diffeomorphism by genus and each such surface admits a metric of constant Gauss curvature. We then immediately see that if \(M\) admits a metric of strictly negative sectional curvature, then it is hyperbolic (admits a constant negative sectional curvature metric), and likewise if \(M\) admits a metric of strictly positive sectional curvature the it is elliptic (admits a constant positive sectional curvature metric). More generally, hyperbolic surfaces are precisely those surfaces admitting a metric with negative average Gauss curvature and similarly for elliptic and parabolic (admits a constant zero sectional curvature metric).

Here then we see that the hyperbolic surfaces, with infinitely many diffeomorphism classes comprise the largest class of closed surfaces, and indeed there is only one class in positive and zero curvature respectively. These hyperbolic surfaces are precisely those presented as \(\H^2/\Gamma\) where \(\Gamma\) is a Fuchsian subgroup of the isometry group \(G_{\H} = \text{PSL}(2, \R)\) of \(\H^2\). Since all compact surfaces \(M\) with metrics \(g\) of strictly negative sectional curvature admit a constant curvature metric, all such surfaces are topologically quotients \(\H^2/\Gamma\). The lifted metric \(\tilde{g} = \pi^{\ast} g\) is then invariant under \(\Gamma\). Thus the deck transformation group of the Riemannian cover \((\H^2, \tilde{g}) \to (N = \H^2/\Gamma, g)\) is precisely the Fuchsian subgroup \(\Gamma\) which is a subgroup of \(G_{\H}\). That is, the constant curvature metric is invariant under the deck transformation group \(\Gamma\) as in \Cref{lem:const_neg} and hence the homotopy \(\tilde{h}(t) = t \tilde{g} + (1-t)g_{\H}\) descends to the quotient \(M = \tilde{M}/\Gamma\) giving a homotopy \(h(t) = t g + (1-t)g_{\H}\) from \(g\) to a constant curvature metric \(g_{\H}\).

\subsection{Thurston's Geometrisation of Three Manifolds}
\label{sec:geometrisation_3d}

The three dimensional case is more complicated than the two dimensional case, but is almost completely understood thanks to Perleman's successful completion \cite{2003math......7245P,2003math......3109P,2002math.....11159P} of Hamilton's program based on the Ricci flow \cite{Hamilton:/1982} to solve the Poincar\'e and Thurston geometrisation conjectures \cite{MR648524}. The remaining piece of the puzzle is the structure of hyperbolic, closed three manifolds. We include here a brief description and refer the reader to \cite{MR705527} and \cite{MR1435975} for in depth discussions of geometrisation and \cite{MR3186136,MR2334563,MR2460872} for expositions of the Hamilton-Perelman proof. Unless explicitly stated otherwise, the results described here may be found in these references.

The geometrisation conjecture may be stated as follows:

\begin{thm}[Thurston Geometrisation]
Every closed three manifold decomposes as a connected sum of prime manifolds, each of which may be cut along tori so that the interior of the resulting manifolds each admits a unique geometric structure of with finite volume from among a possible eight types.
\end{thm}

A prime manifold is simply a manifold that cannot be written as a non-trivial connected sum. The decomposition into prime manifolds was given by \cite{MR0142125}. To say that \(M\) admits a geometry is to say that \(M\) is covered by the geometry, and the classification into eight types of finite volume geometric structures was given by Thurston. Finally the remaining part of the theorem, that such a decomposition of prime manifolds exists was proven by Hamilton and Perleman using the Ricci flow. The eight geometries are
\[
\R^3, \S^3, \H^3, \S^2 \times \R, \H^2 \times \R, \widetilde{SL}_2(\R), \operatorname{Nil}, \operatorname{Solv}.
\]

In the case of \(\operatorname{Solv}\) these are precisely torus and Klein bottle bundles over \(S^1\) or the union of two twisted \(1\)-bundles over the torus or Klein bottle. The remaining six non-hyperbolic geometries are all Seifert Fibre bundles, completely determined by the Euler characteristic of the base space, \(\chi\) and the Euler number of the bundle, \(e\).

The only remaining case then is the hyperbolic case and this has yet has no classification. Of particular relevance here is the question of whether each negatively curved metric is homotopic to a constant negative curvature metric. Note that by the Mostow rigidity theorem \cite{MR0236383}, hyperbolic structures are classified by fundamental group. That is, if the fundamental group \(\pi_1(M_1)\) of a closed hyperbolic manifold, \((M_1, g_1)\) is isomorphic to a \(\pi_1(M_2)\) for another hyperbolic manifold \((M_2, g_2)\), then in fact \((M_1, g_1)\) and \((M_2, g_2)\) are isometric. Equivalently, any homotopy equivalence of hyperbolic manifolds may in fact be homotopied to an isometry.

If \Cref{conj:chow_hamilton} is true so that the integrability condition of \Cref{thm:intg_const_curv} could be removed then arbitrary negatively curved, closed manifolds could be deformed by the Cross Curvature Flow to a hyperbolic manifold. Then every negatively curved, closed three manifold would be smoothly homotopic to a unique isometry class of hyperbolic manifolds, classified by fundamental group.

\section{The Cross Curvature Flow}
\label{sec:xcf}
\subsection{Definition And Basic Properties Of The Flow}
\label{subsec:xcf_defn}

Let \((M, g)\) be a closed, Riemannian manifold and define
\eq{\adj\Ein(X,Y)=g(X,\adj\opEin(Y)),}
where $\adj$ is the adjugate of an endomorphism. The Cross Curvature Flow (in short, XCF) is the evolution equation,
\begin{equation}
\label{eq:xcf}
\begin{cases}
\partial_t g_t  &= 2 \adj\Ein(g_t), \\
g_0 &= g
\end{cases}
\end{equation}
where \(g_0\) has negative sectional curvature. When \(g_0\) has positive sectional curvature, we take instead \(\partial_t g = -2\adj\Ein\) though in this article we will not be concerned with this case.
\begin{rem}
Let \(\pi : \tilde{M} \to M\) be the universal cover, and \(\tilde{g}_t = \pi^{\ast} g_t\). Similarly to the proof of \Cref{thm:intg_const_curv}, we then have
\[
\partial_t \tilde{g}_t = \pi^{\ast} \partial_t \tilde{g}_t = \pi^{\ast} \adj\Ein(g_t) = \adj\Ein(\tilde{g}_t)
\]
and \(\tilde{g}_t\) solves the XCF with initial condition \(\tilde{g}_0 = \pi^{\ast} g_0\). Conversely, if \(\tilde{g}_t\) is a \(G\)-invariant solution of the XCF on \(\tilde{M}\), then there is a unique solution, \(g_t\) of the XCF on \(M\) such that \(\tilde{g}_t = \pi^{\ast} g_t\).
\end{rem}

The definition here makes sense in any dimension. In three dimensions \(g_t\) has negative sectional curvature if and only if \(\Ein\) is positive definite. In this case we may also write
\[
\adj\Ein = \det \opEin \Ein^{-1}.
\]
In an orthonormal basis of eigenvectors \(E_1, E_2, E_3\) for \(\opEin\), with eigenvalues \(\lambda_1, \lambda_2, \lambda_3\), we have for distinct indices, \(i, k, \ell\),
\[
\adj\opEin (E_i) = \det\opEin \opEin^{-1}(E_i) = \lambda_i \lambda_k \lambda_{\ell} \frac{1}{\lambda_i} E_i = \lambda_k \lambda_{\ell} E_i
\]
where \(i,k,\ell\) are distinct indices. Thus
\[
\adj\Ein(E_i, E_j) = g(\lambda_k \lambda_{\ell} E_i, E_j) = \lambda_k \lambda_{\ell} \delta_{ij}.
\]
The tensor \(\adj\Ein\) is referred to as the \emph{cross curvature tensor}. The origin of the name is that the \(i\)'th eigenvalue of \(\adj\Ein\) is the ``cross terms'' \(\lambda_k \lambda_{\ell}\) of the remaining eigenvalues.

There is an equivalent ways to write \(\adj\Ein\) in three dimensions. In fact, both these definitions make sense in any dimension, however it is only in three dimensions that they coincide.
\begin{lemma}
\label{lem:xcf_equiv}

In three dimensions, we have
\[
\adj\Ein(X, Y) = -\frac{1}{2} \Ric_{\Ein} (X, Y) :=- \frac{1}{2} \Tr \br{Z \mapsto \Rm(\opEin(Z), X) Y}
\]
\end{lemma}
\begin{proof}
As noted above,
\[
\adj\Ein(E_i, E_j) = \lambda_k \lambda_{\ell} \delta_{ij}.
\]
There holds
\[
\begin{split}
\Ric_{\Ein}(E_i, E_j) &= \sum_{m=1}^3 \Rm(\opEin(E_m), E_i, E_j, E_m) \\
&= \sum_{m=1}^3 \lambda_m \Rm(E_m, E_i, E_j, E_m) \\
&= -\sum_{m=1}^3 \lambda_m \hat{\lambda}_{mi}\delta_{ij}.
\end{split}
\]
where \(\hat{\lambda}_{mi} = \lambda_k\) if \(m,i,k\) are distinct indices and is zero if \(m=i\). Now, if \(i = j\), and \(i, k, \ell\) are distinct indices, the sum is over \(m=k, \ell\) giving
\[
\Ric_{\Ein}(E_i, E_j) = -\left(\lambda_k \lambda_{\ell} + \lambda_{\ell} \lambda_k\right) = -2\lambda_k \lambda_{\ell}.
\]
Hence
\[
\frac{1}{2} \Ric_{\Ein}(E_i, E_j) =  -\lambda_k \lambda_{\ell} \delta_{ij} = -\adj\Ein(E_i, E_j).
\]
\end{proof}

\begin{rem}
In \cite[Lemma 3]{MR2055396} and \cite[Equation (3)]{MR2207496}, essentially the same result is obtained by contracting with the measure \(\mu\).
\end{rem}

In the case of integrable (and hence isometrically embeddable) solutions of XCF, we have the following observation of Ben Andrews.

\begin{lemma}[{\cite[Section 11]{MR3344442}}]
\label{lem:xcf_gcf}
The induced metric under the Gauss Curvature Flow of convex, spacelike, co-compact hypersurfaces in Minkowski space evolves by XCF.
\end{lemma}

\begin{proof}
The Gauss Curvature Flow of hypersurfaces in Minkowksi space is the evolution equation
\[
\partial_t F = K\nu
\]
where \(K = \det \W\) is the Gauss curvature. Under this equation the metric evolves by
\[
\partial_t g = 2KA.
\]
\Cref{lem:ein_W} gives \(A = \sqrt{\det \opEin} \Ein^{-1}\) and
\[
K = \det \W = \det (\sqrt{\det \opEin} \opEin^{-1}) = (\det \opEin)^{3/2} (\det \opEin)^{-1} = \sqrt{\det \opEin}
\]
so that
\[
\partial_t g = 2KA = 2 \det \opEin \Ein^{-1} = 2 \adj \Ein.
\]
\end{proof}

\subsection{Short Time Existence And Uniqueness}
\label{subsec:xcf_existence_uniqueness}

The question of short time existence and uniqueness of solutions to geometric, parabolic equations on tensor bundles is complicated by the diffeomorphism invariance of the problem leading to degeneracies in the principal symbol. As a result the flow is degenerate parabolic and the standard theory must be modified to take this into account. For the Ricci flow, \cite{Hamilton:/1982} dealt with this issue by employing the Nash-Moser inverse function theorem (see for example \cite{MR656198}) and similarly for the XCF \cite{MR2055396}. However, as pointed out by \cite{MR2207496}, whilst this method is well suited to the Ricci flow which enjoys a first order integrability condition, the integrability condition for the XCF is second order. The question of whether the technique in \cite{Hamilton:/1982} applies is certainly of interest, remaining (as far as we know) an open question in general.

But there is another approach to short time existence and uniqueness known as the DeTurck Trick, that also has the advantage of simplicity in comparison with the Nash-Moser approach. For the Ricci flow, \cite{MR697987} gave an alternative proof to short time existence and uniqueness by breaking the diffeomorphism invariance (i.e. fixing a gauge) to obtain an equivalent, strictly parabolic flow referred to as the DeTurck flow. In \cite[Section 6]{MR1375255} a further simplification of DeTurck's method was given. \cite{MR2207496} adapted this approach to the XCF to easily obtain existence. However, he is omitting the proof of uniqueness and to the opinion of the authors it remained a bit unclear why Buckland's gauge fixing leads to a valid argument for uniqueness. Below we will fill in this detail including its relation to the harmonic map heat flow as in \cite{MR1375255}. Anyhow also XCF enjoys the following result.
\begin{thm}[{\cite[Lemma 4]{MR2055396}}; {\cite[Theorem 1]{MR2207496}}]
\label{thm:xcf_existence_uniqueness}
Given any initial smooth metric \(g\) of negative sectional curvature on a closed three-manifold \(N\), there exists a unique solution \(g_t\) to the XCF,
\[
\begin{cases}
\partial_t g_t &= 2\adj\Ein(g_t) \\
g_0 &= g.
\end{cases}
\]
defined on a maximal time interval \([0, T)\) for some \(T > 0\) or \(T = \infty\).
\end{thm}

To prove the theorem, we need the principal symbol of the cross curvature tensor. Recall that for a nonlinear, second order differential operator \(D : E \to F\) acting on vector bundles \(E, F\) over \(M\), we define
\begin{equation}
\label{eq:symbol}
\sigma_{\xi} [D_s] = \sigma_{\xi} [D'_s]
\end{equation}
where \(s \in \Gamma(E)\) is a section of \(E\) and
\[
D'_s (u) = \partial_w|_{w=0} D(s + w u)
\]
is the linearisation of \(D\) around \(s\) acting on sections \(u \in \Gamma(E)\). Recall that the principal symbol, \(\sigma_{\xi} [D'_s]\) of \(D'_s\) is obtained by replacing second order derivatives in \(D'_s (u)\) by components of \(\xi\).

Recall that \(\Ein\) is positive definite when \(g\) has negative sectional curvature, hence \(\Ein\) is itself a metric. We may thus raise and lower indices using \(\Ein\) as well as take traces with \(\Ein\). However, rather than using \(\Ein\) directly to raise indices we use the metric raising of \(\Ein^{\sharp}\) defined on one-forms by \(\Ein^{\sharp} (\alpha, \beta) = E(\alpha^{\sharp}, \beta^{\sharp})\). Using the equivalent formulation of the Cross Curvature Tensor in \Cref{lem:xcf_equiv}, a straightforward computation yields:

\begin{lemma}
\label{lem:xcf_symbol}
The principal symbol of the cross curvature tensor is,
\[
\sigma_{\xi} [\adj\Ein_g] (V) = \abs{\xi}^2_{\Ein}V - 2 \Sym \xi \otimes V(\sharp_{\Ein}\xi, \cdot) + \Tr_{\Ein} V \xi \otimes \xi.
\]
\end{lemma}
\begin{proof}
See \cite[Lemma 4]{MR2055396} and \cite[Theorem 1]{MR2207496}.
\end{proof}

\begin{proof}[Proof of \Cref{thm:xcf_existence_uniqueness}]
{\color{red} Indicate why the standard harmonic map heat flow approach works to prove Thm. 3.4}
\end{proof}

\subsection{Identities For Non Integrable Solutions}
\label{subsec:xcf_nonintg}

The general, non integrable case poses a number of difficulties. At the heart of these difficulties is the \emph{Devil tensor} \eqref{eq:devil} that vanishes if and only if the solution is integrable (\Cref{lem:cubicform_codazzi}).

We start by defining \(V_{ij}=(\Ein^{-1})^{kl}g_{ik}g_{lj}\)
and
\begin{align}
u&=\log\det\Ein,\quad
\Ob_{ij}=\sqrt{\det\Ein}V_{ij},\\
\T^{kij}&=\Ein^{kl}\nabla_l \Ein^{ij},\quad \T^i=V_{jk}\T^{ijk}=\Ein^{ij}\nabla_ju.
\end{align}
We have the following \(O(3)\) irreducible decomposition of $\T^{ijk}:$
\begin{align}
\T^{ijk}-\T^{jik}&=\mathrm{L}^{ijk}-\mathrm{L}^{jik}+\frac{1}{2}\left(\T^i\Ein^{jk}-\T^j\Ein^{ik}\right)
\end{align}
where the traces of $\mathrm{L}^{ijk}$ with respect to $V_{kl}$ are zero.

The Devil tensor is defined by
\begin{equation}
\label{eq:devil}
\Dv^{ijk}=\mathrm{L}^{ijk}-\mathrm{L}^{jik}
\end{equation}
and it satisfies interesting identities such as
\begin{align}
\Dv^{ijk}=&-\Dv^{jik},\quad \Dv^{ijk}+\Dv^{kij}+\Dv^{jki}=0,\\
V_{ij}\Dv^{ijk}=&V_{ik}\Dv^{ijk}=V_{jk}\Dv^{ijk}=0,\\
V_{ij}\nabla_k\Dv^{kij}=&\frac{1}{2}|\Dv^{ijk}|_V^2.
\end{align}

\begin{lemma}
\label{lem:cubicform_codazzi}

We have the following identity.
\begin{align}
|\Dv^{ijk}|_V^2=\frac{1}{\det\Ein}|\nabla_i\Ob_{jk}-\nabla_j\Ob_{ik}|_{\Ein}^2.
\end{align}
In particular, \(\Dv\equiv 0\) if and only if \(\Ob\) is Codazzi.
\end{lemma}

\begin{proof}
Using the definition we calculate
\begin{align}
\T^{kij}=&-\Ein^{kl}\Ein^{im}\Ein^{jn}\nabla_l V_{mn}=-\Ein^{kl}\Ein^{im}\Ein^{jn}\nabla_l (e^{-\frac{1}{2}u}\Ob_{mn})\\
=&-\frac{1}{\sqrt{\det\Ein}}\Ein^{kl}\Ein^{im}\Ein^{jn}\nabla_l \Ob_{mn}+\frac{1}{2}\T^k\Ein^{ij}.
\end{align}
Thus we obtain
\begin{align*}
\T^{kij}-\T^{ikj}=&\frac{\Ein^{kl}\Ein^{im}\Ein^{jn}}{\sqrt{\det\Ein}}\left(\nabla_m \Ob_{ln}-\nabla_l \Ob_{mn}\right)+\frac{1}{2}(\T^k\Ein^{ij}-\T^i\Ein^{kj}).
\end{align*}
Hence we have the identities
\begin{align}
\Dv^{kij}&=\frac{\Ein^{kl}\Ein^{im}\Ein^{jn}}{\sqrt{\det\Ein}}\left(\nabla_m \Ob_{ln}-\nabla_l \Ob_{mn}\right)\\
|\Dv^{ijk}|_V^2\det\Ein&=|\nabla_i\Ob_{jk}-\nabla_j\Ob_{ik}|_{\Ein}^2.
\end{align}
\end{proof}

\subsection{Towards Hyperbolic Convergence}
\label{subsec:xcf_hyperbolic_convergence}

The following results support \Cref{conj:chow_hamilton} that the XCF deforms arbitrary negatively curved metrics to a hyperbolic metric.

Under the XCF, it is not so easy to prove directly that negative sectional curvature is preserved. However, it is possible to prove that if \(\det \Ein \to 0\), then a singularity must occur. This is the basis of the following:

\begin{prop}[{\cite[p. 8]{MR2055396}}]
Let $T$ be the maximal time of smooth existence of XCF. Then \(\det \Ein > 0\) on \([0, T)\) and hence the sectional curvatures remain negative as long as the solution is smooth.
\end{prop}

The following monotonicity property supports \Cref{conj:chow_hamilton} that the flow should evolve negative curvature metrics to a hyperbolic metric. Note that by the arithmetic-geometric mean inequality applied to the eigenvalues of \(\Ein\), \(\tfrac{1}{3} \Tr_g \Ein \geq (\det \Ein)^{1/3}\) with equality if and only if \(\Ein = \lambda g\) for some smooth function \(\lambda\). Since \(\dive_g \Ein = \dive_g g = 0\), we must then also have \(\lambda \equiv \text{ const.}\) and hence equality occurs if and only if \(g\) has constant sectional curvature.

\begin{thm}[{\cite[Theorem 8]{MR2055396}}]
\label{thm:hyperbolicity}
Under the XCF
\[J(M_t):=\int \frac{\operatorname{Tr}_g\Ein}{3}- (\det\Ein)^{\frac{1}{3}}d\mu\]
is non-increasing. Moreover, \(\frac{d}{dt}J(M_t) = 0\) if and only if $g$ has constant curvature.
\end{thm}

\begin{thm}[\cite{MR2448593}]
\label{thm:hyperbolic_stability}

Any constant curvature metric is asymptotically stable under the XCF after suitable normalisation.
\end{thm}

\subsection{Harnack inequality And Solitons}
\label{subsec:xcf_harnack_solitons}

Differential Li-Yau-Hamilton Harnack inequalities have proved an indispensable tool in the study of curvature flows. For the XCF, \cite[p. 9]{MR2055396} remarked that it is hoped such an inequality holds. For integrable solutions of the XCF this is true. In general, the non-vanishing of the Devil Tensor \eqref{eq:devil} causes significant difficulties in obtaining a Harnack inequality. The evolution of the devil tensor is very complicated and it's not clear whether or not it is amenable to the  maximum principle.

\begin{thm}(\cite[Section 6]{BIS4})
\label{thm:harnack}
The following Li-Yau-Hamilton Harnack inequality holds for integrable solutions to the XCF,
\[
\partial_t \sqrt{\det\Ein} - \frac{1}{\sqrt{\det\Ein}} \abs{\nabla \sqrt{\det\Ein}}_{\Ein}^2 + \frac{3}{4t}\sqrt{\det\Ein} \geq 0.
\]
\end{thm}

Solitons are closely related to the Harnack inequality. They are fixed points of the flow modulo scaling and diffeomorphism. As such, they are the expected limits (up to rescaling) of the flow. Unlike other flows such as the Ricci Flow, solitons for the XCF are very rigid.

\begin{thm}
The only solitons to the XCF are constant curvature metrics.
\end{thm}

\begin{proof}
By definition, a solution to the XCF is a soliton if there exists a vector field $W$ and $\lambda\in \mathbb{R}$ such that at some time
\begin{equation}\label{soliton 0}
2\lambda g_{ij}=2\sigma_{ij}+\nabla_iW_j+\nabla_jW_i,
\end{equation}
where \begin{equation}
\sigma_{ij}:=-\frac{1}{2}\Ein^{kl}\Rm_{ikjl}
\end{equation}
is the Cross Curvature Tensor by \Cref{lem:xcf_equiv}.

For an expanding soliton, (\ref{soliton 0}) holds with $\lambda=\frac{1}{4t}.$ By straightforward calculations
\begin{align}\label{trace}
\lambda \Tr_g \Ein=3\det\Ein+\Ein^{ij}\nabla_iW_j.
\end{align}
and
\begin{align}\label{equ00}
\lambda \nabla_l \Tr_g \Ein=3\nabla_l\det\Ein+\nabla_l \Ein^{ij}\nabla_iW_j+\Ein^{ij}\nabla^2_{i,l}W_j+2\sigma_l^mW_m.
\end{align}
Now using
\begin{align}\label{eq: soliton derv}
2\nabla_l\sigma_{ij}+\nabla^2_{l,i}W_j+\nabla^2_{l,j}W_i=0
\end{align}
we derive
\begin{align}
\lambda \nabla_l \Tr_g \Ein&=3\nabla_l\det\Ein+(\lambda g_{ij}-\sigma_{ij})\nabla_l \Ein^{ij}+\Ein^{ij}\nabla^2_{i,l}W_j+2\sigma_l^mW_m\nonumber\\
&=3\nabla_l\det\Ein+\lambda \nabla_l \Tr_g \Ein-\sigma_{ij}\nabla_l \Ein^{ij}+\Ein^{ij}(-2\nabla_i\sigma_{lj}-\nabla^2_{i,j}W_l)+2\sigma_l^mW_m.
\end{align}
Dividing both sides by $\det\Ein$ yields
\begin{align}
3\nabla_lu-V_{ij}\nabla_l \Ein^{ij}-\frac{\Ein^{ij}}{\det\Ein}\nabla_l\sigma_{ij}-\frac{1}{\det\Ein}(\Box W_l-2\sigma_l^kW_k)=0,
\end{align}
where \(\Box:=\Ein^{ij}\nabla^2_{i,j}.\)
Now in view of the identities
\begin{equation}\label{eq0}
\frac{\Ein^{ij}}{\det\Ein}\nabla_l\sigma_{ij}=2\nabla_lu,~\mbox{and}~ V_{ij}\nabla_l\Ein^{ij}=\nabla_lu
\end{equation}
we arrive at
\begin{align}
\Box W_l-2\sigma_l^kW_k=0\Rightarrow W^l\Box W_l-2\sigma_l^kW_kW^l=0.
\end{align}
The second identity reads
\begin{align}\label{soliton equation for W}
\Box \frac{1}{2}|W|^2-\Ein^{kl}g^{ij}\nabla_kW_i\nabla_lW_j-2\sigma^{kl}W_kW_l=0.
\end{align}
By \cite[Lemma 1]{MR2055396}, $\nabla_i\Ein^{ij}=0.$ Therefore, $\int \Box fd\mu=0$ for any smooth function $f$ on $M$.
Now integrating (\ref{soliton equation for W}) against $d\mu$ and taking into account that $\Ein_{ij}$ and $\sigma_{ij}$ are both positive definite proves that $W\equiv0$. Hence, by (\ref{soliton 0}),
\[(\det\Ein)^2=\frac{\det \sigma_{ij}}{\det g_{ij}}=\lambda ^3\Rightarrow \det\Ein=\lambda^{\frac{3}{2}}.\]
Moreover, in view of (\ref{trace}), we have $3\det\Ein=\lambda \Tr_g \Ein$ and finally \[3(\det\Ein)^{\frac{1}{3}}=\Tr_g \Ein.\] Thus the metric has constant negative curvature.
\end{proof}

\subsection{Monotonicity Of Einstein Volume}
\label{subsec:xcf_volume}

Since the Einstein tensor is a metric, it induces a volume form. By \cite[Proposition 9]{MR2055396}, the Einstein volume is monotone non-decreasing along the XCF. We may strengthen this result, characterising solutions to the XCF such that \(\frac{d}{dt} I(M_t) = 0\) as precisely the integrable solutions, a question posed in \cite{MR2055396}, page 9.

\begin{thm}
\label{thm:volume_monotonicity}
Under the XCF of negative sectional curvature, the \emph{Einstein Volume},
\[
I(M_t):=\int \sqrt{\det\Ein}\,d\mu
\]
is non-decreasing. Moreover, \(\frac{d}{dt} I(M_t)= 0\) if and only if \(\Ob\) is Codazzi if and only if the Riemannian universal cover \((\tilde{M}, \tilde{g})\) embeds isometrically into Minkowski space \(\R^{3,1}\) as a locally convex, co-compact, spacelike hypersurface.
\end{thm}

\begin{proof}
By \cite[Proposition 9]{MR2055396}, we have
\begin{align}
(\partial_t-\Box)(\det\Ein)^\eta=(\frac{\eta}{2}|\Dv^{ijk}|_V^2+\frac{\eta}{2}(1-2\eta)|\T^i|^2_V-2\eta\mathrm{H})(\det\Ein)^\eta,
\end{align}
where $\mathrm{H}=\operatorname{Tr}_g\sigma.$ Therefore, we have the identity
\begin{align}
\frac{d}{dt}\int(\det\Ein)^\eta d\mu=\int (\frac{\eta}{2}|\Dv^{ijk}|_V^2+\frac{\eta}{2}(1-2\eta)|\T^i|^2_V+(1-2\eta)\mathrm{H})(\det\Ein)^\eta d\mu.
\end{align}
In particular, for $\eta=\frac{1}{2}$ we obtain
\begin{align}
\frac{d}{dt}I(M_t)=\frac{1}{4}\int |\Dv^{ijk}|_V^2\sqrt{\det\Ein}d\mu
\end{align}
and monotonicity follows. Now apply \Cref{lem:cubicform_codazzi} and \Cref{thm:intg_embed} to obtain the second part.
\end{proof}

\section*{Acknowledgments}

Parts of this work were written while JS was enjoying the hospitality of the Department of Mathematics at Columbia University in New York, a visit which is funded by the "Deutsche Forschungsgemeinschaft" (DFG, German research foundation) within the research scholarship "Quermassintegral preserving local curvature flows", grant number SCHE 1879/3-1. JS would like to thank the DFG, Columbia University and especially Prof. Simon Brendle for their support. PB was supported by the ARC within the research grant “Analysis of fully non-linear geometric problems and differential equations”, number DE180100110.


\printbibliography

\end{document}
